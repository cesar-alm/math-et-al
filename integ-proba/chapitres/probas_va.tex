\documentclass[../integ-proba.tex]{subfiles}

\begin{document}

    \chapter{Variables aléatoires}

    Voyons maintenant ce qu'est une variable aléatoire.
    C'est une appellation étrange, puisqu'une variable aléatoire n'est en fait ni une variable, ni quelque-chose d'aléatoire.
    En réalité, il s'agit simplement d'une fonction qui possède certaines propriétés (notamment la mesurabilité).
    En général, nous nous limiterons ici à l'étude des variables aléatoire réelles, des vecteurs aléatoires et des variables aléatoires discrètes.

    \section{Définitions}

    Dans toute cette section, on se donne un espace probabilisé $\left(\Omega, \A, \Pb\right)$.
    On rappelle que $\B(E)$ désigne la tribu des boréliens de l'espace topologique $E$.

    \begin{defi}
        Une \textbf{variable aléatoire réelle} est une fonction mesurable
        \begin{displaymath}
            X:\left(\Omega, \A, \Pb\right) \longrightarrow \left(\R, \B(\R)\right)
        \end{displaymath}
    \end{defi}

    \begin{defi}
        Un \textbf{vecteur aléatoire} est une fonction mesurable
        \begin{displaymath}
          X:\left(\Omega, \A, \Pb\right) \longrightarrow \left(\R^n, \B(\R^n)\right)
        \end{displaymath}
    \end{defi}

    \begin{defi}
        \label{defi:variables_aleatoires_discretes}
        Une \textbf{variable aléatoire discrète} est une fonction mesurable
        \begin{displaymath}
          X:\left(\Omega, \A, \Pb\right) \longrightarrow \left(\N, \mathcal{P}(\N)\right)
        \end{displaymath}
    \end{defi}

    On peut également introduire la définition suivante, qui ne sera pas utile en pratique, mais qui a le mérite d'unifier les trois définitions précédentes :

    \begin{defi}
        Plus généralement, si $E$ est un espace topologique quelconque, une \textbf{variable aléatoire} est une fonction mesurable
        \begin{displaymath}
          X:\left(\Omega, \A, \Pb\right) \longrightarrow \left(E, \B(E)\right)
        \end{displaymath}
    \end{defi}

    \begin{rem}
        On remarquera que, exceptés l'espace d'arrivée, ces définitions sont identiques : dans tous les cas, on choisit des fonctions mesurables, et conformément à la simplification expliquée à la remarque~\ref{rem:simplification_notation_fonctions_mesurables}, on choisit la tribu borélienne sur l'espace d'arrivée.
        Cela peut ne pas sembler évident pour la définition~\ref{defi:variables_aleatoires_discretes} concernant les variables aléatoires discrètes : on rappelle donc que les boréliens de $\N$ (muni de la topologie discrète) sont exactement les parties de $\N$ (pour plus de détails, se référer à l'exemple~\ref{exemple:boreliens_et_topologie_discrete}).
    \end{rem}

    \begin{rem}
        On fera attention au fait que :
        \begin{itemize}
            \itemb une variable aléatoire est une fonction qui prend ses valeurs dans $\Omega$ ;
            \itemb une mesure de probabilité est une fonction qui prend ses valeurs dans $\A$.
        \end{itemize}
        Ainsi, si $\omega \in \Omega$ :
        \begin{itemize}
            \itemb écrire $X(\omega)$ a du sens.
            \itemb écrire $\Pb(\omega)$ n'a pas de sens ($\omega$ n'a aucune chance d'être dans $\A$, puisque ce n'est même pas une partie de $\Omega$).
            \itemb écrire $\Pb(\{\omega\})$ a du sens dès que $\{\omega\}\in \A$ (\textit{ie} dès que $\{\omega\}$ est mesurable).
            \itemb écrire $X(\{\omega\})$ n'a pas de sens.
        \end{itemize}
    \end{rem}

    \begin{prop}
        \label{prop:composition_var_bor}
        Soit $E$ et $F$ deux espaces topologiques.
        Soit $X:\Omega \longrightarrow E$ une variable aléatoire à valeurs dans $E$.
        Soit $g:E \longrightarrow F$ une fonction borélienne.
        Alors $g \circ X$ est une variable aléatoire à valeurs dans $F$.
    \end{prop}

    \begin{rem}
        Cette proposition est une conséquence immédiate du résultat de composition de fonctions mesurables.
    \end{rem}

    \section{Loi d'une variable aléatoire}

    Dans toute cette section, on se donne
    \begin{itemize}
        \itemb un espace probabilisé $\left(\Omega, \A, \Pb\right)$ ;
        \itemb une variable aléatoire quelconque $X$ à valeurs dans un espace topologique $\left(E, \B(E)\right)$.
    \end{itemize}

    Ces définitions peuvent être données pour des variables aléatoires quelconques, mais seront utiles en pratique pour les trois types de variables aléatoires que nous avons vues.

    \begin{defi}
        La \textbf{loi de probabilité de} $X$ $\Pb_X$ est définie comme :
        \begin{displaymath}
            \Pb_X=\Pb \circ X^{-1}
        \end{displaymath}
        C'est la mesure image de $\Pb$ par $X$.
        Il s'agit d'une loi de probabilité sur $\left(E, \B(E) \right)$
    \end{defi}
    % todo parler de mesure image quelque part d'autre

    \begin{rem}
        Ces notations étant lourdes, on préfèrera en introduire de nouvelles, plus simples à interpréter.
        Attention : elles n'ont pas de sens en elles-mêmes, mais ont l'avantage de bien faire comprendre la notion sous-jacente.
        \begin{itemize}
            \itemb Si $F \in \B(E)$, on notera :
            \begin{displaymath}
                    \left\{ X \in F \right\} \coloneqq X^{-1}(F) = \left\{ \omega \in \Omega | X(\omega) \in F \right\} \in \A
            \end{displaymath}
            d'où :
            \begin{displaymath}
                \Pb(X \in F) = \Pb \circ X^{-1}(F) = \Pb(X^{-1}(F)) = \Pb_X(F)
            \end{displaymath}
            \itemb Si $F = \left\{ e \right\}$, alors notera plutôt :
            \begin{displaymath}
                \left\{ X = e \right\} \coloneqq X^{-1}(\left\{ e \right\}) = \left\{ \omega \in \Omega | X(\omega) = e \right\} \in \A
            \end{displaymath}
            d'où :
            \begin{displaymath}
                \Pb(X = e) = \Pb \circ X^{-1}(\left\{ e \right\}) = \Pb(X^{-1}(\left\{ e \right\})) = \Pb_X(\left\{ e \right\})
            \end{displaymath}
            \itemb Si $E=\R$, et que $F = [x,y]$, alors on notera plutôt :
            \begin{displaymath}
                \left\{ x \leq X \leq y \right\} \coloneqq X^{-1}([x,y]) = \left\{ \omega \in \Omega | X(\omega) \in [x,y] \right\} \in \A
            \end{displaymath}
            d'où :
            \begin{displaymath}
                \Pb(x \leq X \leq y) = \Pb \circ X^{-1}([x,y]) = \Pb(X^{-1}([x,y])) = \Pb_X([x,y])
            \end{displaymath}
            \itemb etc\ldots
        \end{itemize}
    \end{rem}

    \begin{rem}
        La fonction de répartition de la loi de $X$ sera abusivement appelée fonction de répartition de $X$, et ne sera pas notée $F_{\Pb_X}$ mais plus simplement $F_X$.
    \end{rem}

    \begin{defi}
        Soit $\Q$ une loi de probabilité sur $\left(E, \B(E) \right)$.
        On dit que $X$ \textbf{suit la loi} $\Q$ lorsque $\Pb_X = \Q$.
    \end{defi}

    \begin{rem}
        Conformément à la remarque~\ref{rem:densite}, et en utilisant le théorème~\ref{thm:cns_densite}, \textbf{si $\Q$ est une loi sur $\R$ qui admet une densité $f$, pour prouver que $X$ suit la loi $\Q$, il suffit de vérifier que $X$ est bien à valeurs dans $\R$, puis de montrer que $\forall x \in \R, \Pb(X \leq x) = \int_{-\infty}^x f(t) \textnormal{d}t$.}
        En effet,
        \begin{displaymath}
            \begin{array}{rcl}
                \Pb(X \leq x) & = & \Pb(X \in ]-\infty, x]) \\
                           & = & \Pb(X^{-1}(]-\infty, x])) \\
                           & = & \Pb_X(]-\infty, x]) \\
                           & = & \int_{-\infty}^x f(t) \textnormal{d}t
            \end{array}
        \end{displaymath}
        Evidemment, cette condition est aussi une condition nécessaire.

        Dans ce cas, \textbf{on dit que $X$ est une variable aléatoire réelle à densité} (\textit{stricto sensu}, on devrait dire que la loi de $X$ est à densité, mais cet abus de language est toléré).
    \end{rem}

    \begin{rem}
        Conformément à la remarque~\ref{rem:discretes}, \textbf{si $\Q$ est une loi discrète sur $E$, pour prouver que $X$ suit la loi $\Q$, il suffit de vérifier que $X$ est bien à valeurs dans $E$ , puis de montrer que $\forall e \in E, \Pb(X = e) = \Q(\left\{ e \right\})$.}
        En effet,
        \begin{displaymath}
            \begin{array}{rcl}
                \Pb(X = e) & = & \Pb(X \in \left\{ e \right\}) \\
                           & = & \Pb(X^{-1}(\left\{ e \right\})) \\
                           & = & \Pb_X(\left\{ e \right\})
            \end{array}
        \end{displaymath}
        Evidemment, cette condition est aussi une condition nécessaire.
    \end{rem}

    \begin{rem}
        Attention, ce n'est pas parce que deux variables aléatoires ont la même loi qu'elles sont égales !
        Voici un contre-exemple.
        Il utilise cependant des notions qui n'ont pas encore été abordées à ce stade.

        Considérons l'espace probabilisé $\Omega \coloneqq \{\text{beau temps}, \text{mauvais temps}\}$ muni de l'ensemble de ses parties, et de la mesure de probabilité $\Pb$ uniforme.
        On a donc :
        \begin{displaymath}
            \Pb(\{\text{beau temps}\}) = \Pb(\{\text{mauvais temps}\}) = \frac{1}{2}
        \end{displaymath}
        Puis posons la variable aléatoire discrète $X$ à valeurs dans $\{0,1\}$, telle que $X(\{\text{beau temps}\})=1$ et $X(\{\text{mauvais temps}\})=0$.

        $X$ suit alors la loi uniforme sur l'espace $\{0,1\}$.

        Considérons maintenant l'espace probabilisé $\Omega' \coloneqq \{\text{temps chaud}, \text{temps froid}\}$ muni de l'ensemble de ses parties, et de la mesure de probabilité $\Pb'$ uniforme.
        On a donc :
        \begin{displaymath}
            \Pb(\{\text{temps chaud}\}) = \Pb(\{\text{temps froid}\}) = \frac{1}{2}
        \end{displaymath}
        Puis posons la variable aléatoire discrète $Y$ à valeurs dans $\{0,1\}$, telle que $Y(\{\text{temps chaud}\})=1$ et $Y(\{\text{temps froid}\})=0$.

        $Y$ suit également la loi uniforme sur l'espace $\{0,1\}$.

        Nous avons donc explicité deux variables aléatoires qui ont même loi, mais qui ne sont pas égales (elles n'ont pas le même espace de départ).
    \end{rem}

    \subsection{Loi jointe et lois marginales}
    % changer la formulation de cette sous-section
    Dans toute cette sous-section, on se donne
    \begin{itemize}
        \itemb un espace probabilisé $\left( \Omega, \A, \Pb \right)$ ;
        \itemb $n$ variables aléatoires quelconques $\left( X_i \right)_{i \in [\![1, n]\!]}$ à valeurs dans $\left( E_i \right)_{i \in [\![1, n]\!]}$.
    \end{itemize}

    Commençons par une définition très générale :

    \begin{defi}
        \label{defi:loi_jointe}
        La \textbf{loi jointe} des $n$ variables aléatoires $\left( X_i \right)_{i \in [\![1, n]\!]}$ est la loi de la variable aléatoire $\left( X_1, \ldots, X_n \right)$ à valeurs dans $\left(E_1, \ldots, E_n\right)$.
        La loi d'une variable aléatoire $X_{i_0}$ est la $i_0$e \textbf{loi marginale de $X$}.
    \end{defi}

    \begin{prop}
        \label{prop:connaissance_jointe}
        La connaissance de la loi jointe permet de connaître la loi marginale.
        La réciproque est fausse.
    \end{prop}

    \begin{proof}
        En effet, si l'on souhaite connaître la $i_0$e loi marginale, il suffit d'écrire :
        \begin{displaymath}
            \begin{array}{rcl}
                \forall A \in \B(E_{i_0}), \Pb_{X_{i_0}}(A) & = & \Pb(X_{i_0}^{-1}(A)) \\
                                                   & = & \Pb(X_1^{-1}(E_1) \cap \ldots \cap X_{i_0}^{-1}(A) \cap \ldots \cap X_n^{-1}(E_n)) \\
                                                   & = & \Pb((X_1, \ldots, X_{i_0}, \ldots, X_n)^{-1}(E_1 \times \ldots \times A \times \ldots \times E_n)) \\
                                                   & = & \Pb_{(X_1, \ldots, X_{i_0}, \ldots, X_n)}(E_1 \times \ldots \times A \times \ldots \times E_n) \\
                                                   & = & \int_{E_1} \ldots \int_{E_{i_0 - 1}} \int_A \int_{E_{i_0 + 1}} \ldots \int_{E_n} \textnormal{d} \Pb_{\left( X_1, \ldots, X_n \right)}(x_1, \ldots, x_n)
            \end{array}
        \end{displaymath}

        Cependant, \textbf{la réciproque est fausse !}
        En effet, la connaissance des lois marginales ne donne aucune information sur la \textit{relation} qu'ont les variables aléatoires entre elles.
        Intuitivement, il est donc impensable de pouvoir connaître la loi du vecteur aléatoire résultant.
        % todo contrexemple
    \end{proof}

    En pratique cependant, on n'emploie les lois jointes et marginales uniquement lorsque l'on traite avec des variables aléatoires réelles ou discrètes.
    Dans la suite de cette section, on se donne donc $n$ variables aléatoires réelles ou discrètes $\left( X_i \right)_{i \in [\![1, n]\!]}$, que l'on peut voir comme un vecteur aléatoire $X$ à $n$ composantes.
    Remarquons que voir ces variables comme un vecteur aléatoire n'était pas possible jusqu'à maintenant, puisqu'elles n'étaient pas forcément à valeurs réelles.

    Ceci étant dit, on énonce de nouveau la définition~\ref{defi:loi_jointe} dans ce cas particulier :

    \begin{defi}
        La \textbf{loi jointe} des $n$ variables aléatoires $\left( X_i \right)_{i \in [\![1, n]\!]}$ est la loi du vecteur aléatoire $X$.
        La loi d'une variable aléatoire $X_{i_0}$ est la $i_0$e \textbf{loi marginale de $X$}.
    \end{defi}

    Maintenant, énonçons de nouveau la proposition~\ref{prop:connaissance_jointe} dans les deux cas particuliers suivants :
    \begin{itemize}
        \itemb $X$ est une variable aléatoire réelle à densité (proposition~\ref{prop:connaissance_jointe_densite});
        \itemb $X$ est une variable aléatoire discrète sur $\N^n$ (proposition~\ref{prop:connaissance_jointe_discrete}).
    \end{itemize}

    \begin{prop}
        \label{prop:connaissance_jointe_densite}
        Supposons que $X$ soit un vecteur aléatoire à densité $f$.
        Alors les lois marginales sont à densité (\textit{ie} les $\left( X_i \right)_{i \in [\![1, n]\!]}$ sont des variables aléatoires à densité), de densité $\left(f_i\right)_{i \in [\![1, n]\!]}$, vérifiant pour tout $i \in [\![1, n]\!]$ :
        \begin{displaymath}
            \forall x_i \in \R, f_i(x_i) = \int_{\mathbb{R}} \cdots \int_{\mathbb{R}} f(x_1, \ldots, x_n) \textnormal{d} (x_1, \ldots, x_{i-1}, x_{i+1}, \ldots, x_n)
        \end{displaymath}
    \end{prop}

    \begin{rem}
        Attention, même si toutes les lois marginales sont à densité (\textit{ie} toutes les variables aléatoires $\left( X_i \right)_{i \in [\![1, n]\!]}$ sont à densité), la loi jointe peut ne pas admettre de densité.
        A nouveau, cela est un exemple que la connaissance des lois marginales ne suffit pas pour connaître la loi jointe.
        % todo example
    \end{rem}

    \begin{prop}
        \label{prop:connaissance_jointe_discrete}
        Supposons que $X$ soit une variable aléatoire discrète sur $\N^n$ (\textit{ie} que toutes les $X_1, \ldots, X_n$ soient toutes des variables aléatoires discrètes sur $\N$).
        Alors la connaissance de la loi jointe permet de connaître les lois marginales de la manière suivante :
        \begin{displaymath}
            \forall i \in [\![1, n]\!], \forall k_i \in \N, \Pb(X_i = k_i) = \sum_{k_1, \ldots, k_{i-1}, k_{i+1}, \ldots, k_n \in \N^{n-1}} \Pb(X_1 = k_1, \ldots, X_n = k_n)
        \end{displaymath}
    \end{prop}

    \section{Variables aléatoires indépendantes}

    Dans toute cette section, on se donne un espace probabilisé $\left( \Omega, \A, \Pb \right)$.

    \subsection{Définitions}

    Commençons par une définition générale, valable pour n'importe-quel couple de variables aléatoires :

    \begin{defi}
        Soit $X$ et $Y$ deux variables aléatoires quelconques à valeurs dans $E$ et $F$ respectivement.
        $X$ et $Y$ sont dites \textbf{indépendantes} lorsque l'une des deux conditions équivalentes suivantes est remplie :
        \begin{itemize}
            \itemb la loi jointe de $X$ et $Y$ $\Pb_{(X,Y)}$ vérifie $\Pb_{(X,Y)} = \Pb_X \otimes \Pb_Y$ ;
            \itemb pour tout $(A,B) \in \B(E) \times \B(F)$, $\left\{ X \in A \right\}$ et $\left\{ Y \in B \right\}$ sont deux évènements indépendants.
        \end{itemize}
    \end{defi}

    Généralisons maintenant en énonçant la définition pour $n$ variables aléatoires quelconques :

    \begin{defi}
        Soit $\left( X_n \right)_{n \in [\![1,n]\!]}$ $n$ variables aléatoires quelconques, à valeurs dans $\left( E_1, \ldots, E_n \right)$.
        Les $\left( X_n \right)_{n \in [\![1,n]\!]}$ sont dites \textbf{mutuellement indépendantes} (ou simplement indépendantes), lorsque l'une des deux conditions équivalentes suivantes est remplie :
        \begin{itemize}
            \itemb la loi jointe des $\left( X_n \right)_{n \in [\![1,n]\!]}$ $\Pb_{(X_1, \ldots, X_n)}$ vérifie $\Pb_{(X_1, \ldots, X_n)} = \bigotimes_{i=1}^n \Pb_{X_i}$ ;
            \itemb pour tout $(A_1, \ldots, A_n) \in \prod_{i = 1}^n \B(E_i)$, $\left\{ X_1 \in A_1 \right\}, \ldots, \left\{ X_n \in A_n \right\}$ sont $n$ évènements indépendants.
        \end{itemize}
        Elles sont dites \textbf{deux à deux indépendantes} lorsque pour tous $\left\{ i,j \right\} \in [\![1,n]\!]^2$ tels que $i \neq j$, $X_i$ et $X_j$ sont deux variables aléatoires indépendantes.
    \end{defi}

    \begin{rem}
        On vérifie sans peine que la deuxième condition est équivalente à :
        \begin{displaymath}
            \forall (A_1, \ldots, A_n) \in \prod_{i = 1}^n \B(E_i), \Pb(X_1 \in A_1, \ldots X_n \in A_n) = \prod_{i = 1}^n \Pb(X_i \in A_i)
        \end{displaymath}
        \textbf{En pratique, pour vérifier l'indépendance de $n$ variables aléatoires, on utilisera cette caractérisation facile}.
        Son défaut est de ne pas faire le parallèle avec l'indépendance des évènements (puisqu'on rappelle que pour vérifier l'indépendance d'un ensemble d'évènements, il faut vérifier l'égalité pour tout sous-ensemble fini de ces évènements).
    \end{rem}

    Enfin, énonçons la définition pour une famille quelconque de variables aléatoires quelconques (en particulier valable pour les suites de variables aléatoires):
    \begin{defi}
        Soit $I$ un ensemble quelconque.
        Soit $\left( X_i \right)_{i \in I}$ des variables aléatoires quelconques, à valeurs dans $\left( E_i\right)_{i \in I}$.
        Les $\left( X_i \right)_{i \in I}$ sont dites \textbf{mutuellement indépendantes} (ou simplement indépendantes), lorsque pour toute sous-partie finie $J \subset I$, les $\left( X_i \right)_{i \in J}$ sont mutuellement indépendantes.

        Elles sont dites \textbf{deux à deux indépendantes} lorsque pour tous $\left\{ i,j \right\} \in I^2$ tels que $i \neq j$, $X_i$ et $X_j$ sont deux variables aléatoires indépendantes.
    \end{defi}

    \begin{prop}
        La mutuelle indépendance implique l'indépendance deux à deux.
        La réciproque est fausse.
    \end{prop}

    % densité ??

    \subsection{Deux résultats importants}

    \begin{thm}[\textbf{Transfert d'indépendance}]
        Soit $I$ un ensemble quelconque.
        Soit $\left( X_i \right)_{i \in I}$ des variables aléatoires quelconques, à valeurs dans $\left( E_i\right)_{i \in I}$.
        Supposons que les $\left( X_i \right)_{i \in I}$ soient mutuellement indépendantes (resp. deux à deux indépendantes).
        Soit $\left( \phi_i \right)_{i \in I}$ des fonctions de $E_i$ dans des ensembles $F_i$, $\B(E_i)/\B(F_i)$-mesurables.
        Alors les variables aléatoires $\left( \phi_i \circ X_i \right)_{i \in I}$, à valeurs dans $F_i$, sont encore mutuellement indépendantes (resp. deux à deux indépendantes).
    \end{thm}

    \begin{thm}[\textbf{Lemme des coalitions}]
        %todo
    \end{thm}

    \section{Moments d'une variable aléatoire réelle ou discrète}

    Dans toute cette section, on se donne
    \begin{itemize}
        \itemb un espace probabilisé $\left(\Omega, \A, \Pb\right)$ ;
        \itemb une variable aléatoire réelle ou discrète $X$.
    \end{itemize}

    On ne pourra pas donner de sens à ces notions dans le cas général d'une variable aléatoire quelconque.
    Nous verrons néanmoins par la suite comment généraliser ces notions aux vecteurs aléatoires.

    \begin{defi}
        Soit $r\in\N$.
        On dit que $X$ \textbf{admet un moment d'ordre} $r$ lorsque $\omega \mapsto X^r(\omega)$ est $\Pb$-intégrable.
    \end{defi}

    \begin{rem}
        On rappelle que l'intégrale de Lebesgue est absolue.
        Ainsi, cette définition et les assertions suivantes sont équivalentes :
        \begin{itemize}
            \itemb $\omega \mapsto \left|X(\omega)\right|^r$ est $\Pb$-intégrable.
            \itemb $\displaystyle \int_\Omega\left|X(\omega)\right|^r\text{d}\Pb(\omega) < +\infty$
        \end{itemize}
    \end{rem}

    \begin{defi}
        Soit $r\in\N$.
        Lorsque $X$ admet un moment d'ordre $r$, on définit son \textbf{moment d'ordre } $r$ comme
        \begin{displaymath}
          m_r \coloneqq \int_\Omega X^r(\omega)\text{d}\Pb(\omega)
        \end{displaymath}
    \end{defi}

    \begin{rem}
        On vérifie aisément, par la relation mesure-intégrale, que toute variable aléatoire admet un moment d'ordre 0, et que celui-ci vaut toujours 1.
    \end{rem}

    \begin{prop}
        \label{prop:moments_cascade}
        Soit $r\in\N$.
        Si $X$ admet un moment d'ordre $r$, alors elle admet un moment d'ordre $k$ pour tout $k\in[\![0,r]\!]$.
        Autrement dit, si $X^r$ est $\Pb$-intégrable, alors $X^k$ est $\Pb$-intégrable pour tout $k\in[\![0,r]\!]$.
    \end{prop}

    \begin{proof}
        Soit $k\in[\![0,r]\!]$.
        Nous avons $\left|X\right|^k \leq \left|X\right|^r+1$, donc par croissance et linéarité de l'intégrale :
        \begin{displaymath}
            \int_\Omega \left|X(\omega)\right|^k \text{d}\Pb(\omega) \leq \int_\Omega \left|X(\omega)\right|^r \text{d}\Pb(\omega) + \int_\Omega \text{d}\Pb(\omega)
        \end{displaymath}

        Comme $X$ admet un moment d'ordre $r$, le premier terme de la somme et fini.
        Puis, d'après la relation mesure-intégrale, le deuxième terme de la somme vaut 1 ($\Pb$ est une mesure de probabilité).

        Le terme de gauche est donc fini, donc $X$ admet un moment d'ordre $k$.
    \end{proof}

    \begin{rem}
        \label{rem:warning_moments}
        Attention, on remarquera que ceci n'est vrai que parce que $\Pb$ est une mesure de probabilité !
        Dans le cas d'une mesure quelconque (par exemple, la mesure de Lebesgue), cette démonstration ne permet pas de conclure (car $\lambda(\R) = +\infty$)\ldots et heureusement, car ce résultat est faux !

        Voici un contre-exemple :
        \begin{displaymath}
          x \mapsto \mathds{1}_{\R_+^*}(x)\frac{1}{x}
        \end{displaymath}
        n'est pas intégrable sur $\R$, alors que son carré l'est.
    \end{rem}

    \begin{defi}
        On dit que $X$ \textbf{admet une espérance} lorsqu'elle admet un moment d'ordre 1 (\textit{ie} lorsque $\omega \mapsto X\left(\omega\right)$ est $\Pb$-intégrable).
        Son \textbf{espérance} est alors définie comme son moment d'ordre 1.
        Elle est notée $\esp(X)$.

        On note $\mathcal{L}^1\left(\Omega, \A, \Pb\right)$ l'ensemble des variables aléatoires qui admettent une espérance (\textit{ie} l'ensemble des variables aléatoires $\Pb$-intégrables).
    \end{defi}

    \begin{rem}
        Par abus de notation, on pourra écrire $\esp(X)=+\infty$ lorsque $X$ n'admet pas d'espérance et qu'elle est positive.
        Ainsi, on pourra écrire $\esp(X)$ dès que $X$ est positive, qu'elle admette une espérance ou non.
        Vue la place privilégiée qu'occupent les fonctions positives dans la théorie d'intégration de Lebesgue, cette notation ne surprend pas.

        Attention cependant, si $X$ n'admet pas d'espérance et n'est pas positive, on s'interdira formellement d'écrire $\esp(X)$ (car l'intégrale sous-jacente n'existe peut-être même pas !).
    \end{rem}

    \begin{rem}
        Ces considérations étant faites, on remarque que :
        \begin{itemize}
            \itemb $X$ admet une espérance ssi $\esp(\left| X \right|) < +\infty$ ;
            \itemb $X$ admet un moment d'ordre $r$ ssi $\esp(\left| X \right|^r) < +\infty$ ;
            \itemb si $X$ admet un moment d'ordre $r$, alors $m_r = \esp(X^r)$.
        \end{itemize}
    \end{rem}

    \subsection{Formule de transfert}

    Dans toute cette section, on se donne
    \begin{itemize}
        \itemb un espace probabilisé $\left(\Omega, \A, \mathbb{P}\right)$ ;
        \itemb un espace topologique quelconque $E$ ;
        \itemb une variable aléatoire quelconque $X:\Omega \longrightarrow E$.
    \end{itemize}

    \begin{thm}[\textbf{Formule de transfert}]
        Soit $g:E \longrightarrow \R$ (resp. $g:E \longrightarrow \N$) une fonction borélienne.
        Par la proposition~\ref{prop:composition_var_bor}, $g \circ X$ est encore une variable aléatoire ;
        c'est d'ailleurs une variable aléatoire réelle (resp.\ une variable aléatoire discrète).
        On sait donc donner un sens à son espérance.

        La variable aléatoire réelle (resp.\ discrète) $g \circ X$ sur $\left(\Omega, \A, \mathbb{P}\right)$ admet une espérance (\textit{ie} $\omega \mapsto (g \circ X)(\omega)$ est $\mathbb{P}$-intégrable) ssi la variable aléatoire réelle (resp. discrète) $g$ sur $\left(E, \B(E), \mathbb{P}_X\right)$ admet une espérance (\textit{ie} $x \mapsto g(x)$ est $\mathbb{P}_X$-intégrable).

        Si c'est le cas, alors on a la \textbf{formule de transfert} :
        \begin{displaymath}
            \esp(g \circ X) = \int_E g(x)\textnormal{d}\mathbb{P}_X(x)
        \end{displaymath}
    \end{thm}

    \begin{rem}
        On comprend pourquoi cette formule porte ce nom, lorsque l'on remplace l'espérance par sa définition, dans les expressions précédentes :
        \begin{displaymath}
            \int_\Omega (g \circ X)(\omega) \text{d} \mathbb{P}(\omega) = \int_E g(x) \text{d} \mathbb{P}_X(x)
        \end{displaymath}
        Cette formule transfère le calcul sur un autre espace, muni d'une autre mesure.
        Pour peu que la mesure $\mathbb{P}_X$ soit facile à utiliser, on a intérêt à utiliser cette nouvelle formulation.
        % faire ref au changement de mesure du chap de la construction de l'intégrale.
    \end{rem}

    Voici quelques applications de ce théorème :

    \begin{exemple}
        Soit $X:\Omega \longrightarrow \R$ une variable aléatoire réelle.
        La fonction identité $\text{Id}:\R \longrightarrow \R$ est continue donc borélienne (voir la proposition~\ref{prop:continue_implique_borelienne}).

        En appliquant le théorème précédent, $X$ admet une espérance ssi $\text{Id}$ est $\mathbb{P}_X$-intégrable.
        Dans ce cas, on aura :
        \begin{displaymath}
            \esp(X) = \esp(\text{Id} \circ X) = \int_\R x\textnormal{d}\mathbb{P}_X(x)
        \end{displaymath}

        Pour peu que l'on sache expliciter $\mathbb{P}_X$, cette nouvelle l'expression de l'espérance nous simplifie grandement le calcul.

        Il est évident que l'on obtient le même résultat avec une variable aléatoire discrète.
    \end{exemple}

    \begin{exemple}
        De même, en choisissant la fonction carrée pour $g$ on obtient (si l'espérance de $X^2$ existe) :
        \begin{displaymath}
            \esp(X^2) = \int_\R x^2\textnormal{d}\mathbb{P}_X(x)
        \end{displaymath}

        Notons qu'en choisissant l'identité pour $g$, mais la variable aléatoire $X^2$, on obtient (à nouveau sous réserve d'existence) :
        \begin{displaymath}
            \esp(X^2) = \int_\mathbb{R} x\textnormal{d}\mathbb{P}_{X^2}(x)
        \end{displaymath}
    \end{exemple}

    \begin{rem}
        \label{rem:important_moments}
        Voici deux conséquences importantes de la formule de transfert :
        \begin{itemize}
            \itemb \textbf{si $X$ est une variable aléatoire réelle de densité $f$, alors $g \circ X$ admet une espérance ssi}
            \begin{displaymath}
                \int_\R \left|g(x)\right| f(x) \textnormal{d}x \left(= \int_\R \left|g(x)\right| \textnormal{d} \Pb_X(x) \right) < + \infty
            \end{displaymath}
            Si c'est le cas, alors :
            \begin{displaymath}
                \esp(g \circ X) = \int_\R g(x) f(x) \textnormal{d}x \left(= \int_\R g(x) \textnormal{d} \Pb_X(x) \right)
            \end{displaymath}
            La démonstration est immédiate à l'aide du théorème~\ref{thm:avantage_densite} et de la formule de transfert.
            Si $X$ était un vecteur aléatoire de densité $f$, le résultat est le même (remplacer $\R$ par $\R^n$).
            \itemb \textbf{si $X$ est une variable aléatoire discrète, alors $g \circ X$ admet une espérance ssi}
            \begin{displaymath}
                \sum_{k=0}^{+\infty} \left|g(k)\right| \Pb(X=k) \left(= \int_\R \left|g(x)\right| \textnormal{d} \Pb_X(x) \right) < + \infty
            \end{displaymath}
            Si c'est le cas, alors :
            \begin{displaymath}
                \esp(g \circ X) = \sum_{k=0}^{+\infty} g(k) \Pb(X=k) \left(= \int_\R g(x) \textnormal{d} \Pb_X(x) \right)
            \end{displaymath}
            En effet, dans le cas discret, $\Pb_X = fc$, où $f(k) = \Pb_X(\left\{ k \right\}) = \Pb(X = k)$ (cf.\ théorème~\ref{thm:avantage_discret}).
            Puis le théorème~\ref{thm:avantage_densite} fournit l'équivalence et l'égalité.
            Enfin, le théorème~\ref{thm:lien_series} fournit l'expression sous forme de somme.

            Remarquons que tout a été donné ici dans le cas de $\N$, mais tout s'énonce de même dans les autres cas.
        \end{itemize}

        Dans le cas où $g = \cdot^r$, on obtient les résultats suivants :
        \begin{itemize}
            \itemb \textbf{si $X$ est une variable aléatoire réelle de densité $f$, alors $X$ admet un moment d'ordre $r$ ssi}
            \begin{displaymath}
                \int_\R \left|x\right|^r f(x) \textnormal{d}x < + \infty
            \end{displaymath}
            Si c'est le cas, alors :
            \begin{displaymath}
                \esp(X) = \int_\R x^r f(x) \textnormal{d}x
            \end{displaymath}
            \itemb \textbf{si $X$ est une variable aléatoire discrète, alors $X$ admet un moment d'ordre $r$ ssi}
            \begin{displaymath}
                \sum_{k=0}^{+\infty} \left|k\right|^r \Pb(X=k) < + \infty
            \end{displaymath}
            Si c'est le cas, alors :
            \begin{displaymath}
                \esp(X) = \sum_{k=0}^{+\infty} k^r \Pb(X=k)
            \end{displaymath}
        \end{itemize}
    \end{rem}

    \begin{rem}
        La formule de transfert ne doit pas être confondue avec le transfert d'indépendance !
    \end{rem}

    \subsection{Indépendance et espérance}

    Cette sous-section vise uniquement à donner un résultat important concernant l'espérance de variables aléatoires, lorsqu'il y a indépendance :

    \begin{thm}
        \label{thm:independance_esperance}
        Soit $\left( \Omega, \A, \Pb \right)$ un espace probabilisé et $X$ et $Y$ deux variables aléatoires quelconques sur cet espace, à valeurs dans deux espaces topologiques $E$ et $F$ respectivement.
        Soit $g:E \times F \rightarrow \R$ une fonction borélienne, telle que $g \circ \left( X, Y \right)$ admette une espérance.
        Alors :
        \begin{displaymath}
            \esp(g \circ (X,Y)) = \int_E \left( \int_F g(x,y) \textnormal(d) \Pb_Y(y)\right) \textnormal{d} \Pb_X(x)
        \end{displaymath}
    \end{thm}

    \begin{rem}
        Si $g$ est la multiplication, alors on a immédiatement :
        \begin{displaymath}
            \esp(XY) = \esp(X)\esp(Y)
        \end{displaymath}
    \end{rem}

    \subsection{Variance et covariance}

    \begin{defi}
        On dit que $X$ \textbf{admet une variance} lorsqu'elle admet un moment d'ordre 2 (\textit{ie} lorsque $\omega \mapsto X^2\left(\omega\right)$ est $\Pb$-intégrable).
        Sa \textbf{variance} est alors définie comme :
        \begin{displaymath}
            \var(X)\coloneqq\esp\left(\left(X - \esp\left(X\right)\right)^2\right)=\esp\left(X^2\right) - \esp\left(X\right)^2
        \end{displaymath}

        On note $\mathcal{L}^2\left(\Omega, \A, \Pb\right)$ l'ensemble des variables aléatoires qui admettent une variance (\textit{ie} l'ensemble des variables aléatoires qui sont de carré $\Pb$-intégrables).
    \end{defi}

    \begin{rem}
        Cette définition impose de faire trois remarques :
        \begin{itemize}
            \itemb le théorème qui fournit l'égalité de ces deux expressions s'appelle \textbf{le théorème de König-Huygens}.
            Sa démonstration est immédiate par linéarité de l'intégrale.
            \itemb d'après la deuxième expression, l'existence de $\var(X)$ est assurée dès que $X$ admet un moment d'ordre 1 et 2.
            Cependant, bien que la définition demande que $X$ admette un moment d'ordre 2, elle ne demande pas qu'elle admette un moment d'ordre 1.
            Nous sommes en fait sauvés par la proposition~\ref{prop:moments_cascade} !
            \itemb d'après la proposition~\ref{prop:moments_cascade}, $\mathcal{L}^2\left(\Omega, \A, \mathbb{P}\right) \subset \mathcal{L}^1\left(\Omega, \A, \mathbb{P}\right)$.
            A nouveau, on rappelle que ceci n'est vrai que parce que $\mathbb{P}$ est une mesure de probabilité (voir la remarque~\ref{rem:warning_moments})
        \end{itemize}
    \end{rem}

    \begin{rem}
        Dans le cas quelconque, on peut exprimer la variance ainsi (à l'aide de la formule de transfert) :
        \begin{displaymath}
            \var(X) = \int_{\R} (x - \esp(X))^2 \text{d}\Pb_X (x) = \int_{\R} x^2 \text{d}\Pb_X (x) - \left( \esp(X) \right)^2
        \end{displaymath}

        Cependant, il existe deux cas particuliers où elle s'exprime plus simplement :
        \begin{itemize}
            \itemb dans le cas où $X$ est à densité, la remarque~\ref{rem:important_moments} simplifie l'expression en :
            \begin{displaymath}
                \var(X) = \int_{\R} (x - \esp(X))^2 f(x) \text{d}x = \int_{\R} x^2 f(x) \text{d} x - \left( \esp(X) \right)^2
            \end{displaymath}
            \itemb dans le cas où $X$ est discrète, la même remarque la simplifie en :
            \begin{displaymath}
                \var(X) = \sum_{k=0}^{+\infty} (k - \esp(X))^2 \Pb(X=k) = \sum_{k=0}^{+\infty} k^2 \Pb(X=k) - \left( \esp(X) \right)^2
            \end{displaymath}
        \end{itemize}
    \end{rem}

    \begin{defi}
        Soit $X$ et $Y$ deux variables aléatoires réelles (ou discrètes) de $\mathcal{L}^2\left( \Omega, \A, \Pb \right)$.
        La variable aléatoire $\left( X - \esp(X) \right)\left( Y - \esp(Y) \right)$ admet une espérance et on appelle \textbf{covariance de $X$ et $Y$} le réel :
        \begin{displaymath}
            \text{Cov}(X, Y) \coloneqq \esp \left( \left( X - \esp(X) \right)\left( Y - \esp(Y) \right) \right) = \esp\left(XY\right) - \esp\left(X\right)\esp\left( Y \right)
        \end{displaymath}
    \end{defi}

    \begin{rem}
        Remarquons que $\text{Cov}\left( X,X \right) = \var\left( X \right)$.
    \end{rem}

    % todo cas particulier réel et discret

    \begin{defi}
        Soit $X$ et $Y$ deux variables aléatoires réelles (ou discrètes) de $\mathcal{L}^2\left( \Omega, \A, \Pb \right)$.
        On dit que $X$ et $Y$ sont \textbf{non-corrélées} lorsque $\text{Cov}\left( X, Y \right) = 0$.
    \end{defi}

    \begin{prop}
        Soit $X$ et $Y$ deux variables aléatoires réelles (ou discrètes).
        Si $X$ et $Y$ sont indépendantes, alors elles sont non-corrélées.
        La réciproque est fausse.
    \end{prop}

    \section{Moments d'un vecteur aléatoire}

    Dans toute cette section, on se donne
    \begin{itemize}
        \itemb un espace probabilisé $\left(\Omega, \A, \Pb\right)$ ;
        \itemb un vecteur aléatoire $X = \left( X_i \right)_{i \in \N}$.
    \end{itemize}

    On ne peut pas directement adapter les définitions précédentes pour les vecteurs aléatoires, et il est nécessaire de mettre en place de nouvelles définitions.
    On ne généralisera que les moments d'ordre 1 et 2.

    \begin{defi}
        \label{def:vecteur_esperance}
        On dit que \textbf{$X$ admet un vecteur espérance} lorsque pour tout $i \in [\![1,n]\!]$, $X_i$ admet une espérance.
        Si c'est le cas, alors le \textbf{vecteur espérance de $X$ existe et est défini comme :}
       \begin{displaymath}
           \esp(X) \coloneqq \left( \esp(X_i) \right)_{i \in [\![1,n]\!]}
       \end{displaymath}
    \end{defi}

    \begin{defi}
        On dit que \textbf{$X$ admet une matrice de covariance} lorsque pour tout $i \in [\![1, n]\!]$, $X_i$ admet une variance.
        Si c'est le cas, alors la \textbf{matrice de covariance de $X$ existe et est définie comme :}
        \begin{displaymath}
            \var(X) \coloneqq \left( \text{Cov}(X_i, X_j) \right)_{\left( i,j \right) \in [\![1, n]\!]^2}
        \end{displaymath}
    \end{defi}

    \begin{rem}
        Notons que sur la diagonale, nous retrouvons $\left( \var (X_i) \right)_{i \in [\![1,n]\!]}$.
    \end{rem}

    \begin{prop}
        La matrice de covariance de $X$ est une \textbf{matrice symétrique positive}.
    \end{prop}

    \section{Conditionnement par une variable aléatoire}

    Les lois conditionnelles permettent de répondre à deux problématiques :
    \begin{itemize}
        \itemb On sait conditionner par un évènement de probabilité non-nulle (cf.\ section~\ref{sec:indep_cond}).
        Cependant, on ne sait pas conditionner par un évènement de probabilité nulle de la forme $\left\{ X = x \right\}$.
        Les lois conditionnelles permettront de donner un sens à une telle notion.
        \itemb Lorsque deux variables aléatoires réelles $X$ et $Y$ sont indépendantes, $\Pb_{(X,Y)} = \Pb_X \otimes \Pb_Y$ donc par le théorème de Fubini, on peut calculer les espérances de fonctions de $X$ et $Y$ très facilement en séparant les variables.
        Lorsqu'elles ne sont pas indépendantes, on énoncera un théorème de Fubini dit \og conditionnel \fg, qui permettra de faire une transformation similaire.
    \end{itemize}

    De plus, comme on mettra en place de nouvelles lois de probabilité (conditionnelles), on pourra également définir de nouvelles espérances (conditionnelles).

    \subsection{Lois conditionnelles}

    Dans toute cette section, on se donne :
    \begin{itemize}
        \itemb un espace probabilisé $\left( \Omega, \A, \Pb \right)$.
    \end{itemize}

    Enonçons le théorème d'existence des lois conditionnelles dans le cas de variables aléatoires réelles :

    \begin{thm}[\textbf{Existence des lois conditionnelles pour des variables aléatoires réelles}]
        Soit $X$ et $Y$ deux variables aléatoires réelles.
        Alors il existe une famille de lois de probabilité sur $\left( \R, \B(\R) \right)$, notée $\left( \Pb_{X | Y = y} \right)_{y \in \R}$, telle que :
        \begin{displaymath}
            \forall \left( A, B \right) \in \B(\R)^2, \Pb_{\left( X,Y \right)}(A \times B) = \int_B \Pb_{X | Y = y}(A) \textnormal{d}\Pb_Y(y)
        \end{displaymath}

        $\Pb_{X | Y = y}$ s'appelle la \textbf{loi conditionnelle de $X$ sachant $Y = y$}.
    \end{thm}

    Enonçons-le maintenant de même pour des vecteurs aléatoires :

    \begin{thm}[\textbf{Existence des lois conditionnelles pour des vecteurs aléatoires}]
        Soit $X$ et $Y$ deux vecteurs aléatoires à valeurs dans $\R^m$ et $\R^n$.
        Alors il existe une famille de lois de probabilité sur $\left( \R^m, \B(\R^m) \right)$, notée $\left( \Pb_{X | Y = y} \right)_{y \in \R^n}$, telle que :
        \begin{displaymath}
            \forall \left( A, B \right) \in \B(\R^m) \times \B(\R^n), \Pb_{\left( X,Y \right)}(A \times B) = \int_B \Pb_{X | Y = y}(A) \textnormal{d}\Pb_Y(y)
        \end{displaymath}

        $\Pb_{X | Y = y}$ s'appelle la \textbf{loi conditionnelle de $X$ sachant $Y = y$}.
    \end{thm}

    \begin{rem}
        \label{rem:warning_cond}
        Attention, cette notation n'a \textit{a priori} rien à voir avec la théorie des probabilités conditionnées par un évènement.
        En effet, si $\left\{ Y = y \right\}$ est de probabilité nulle, la définition habituelle ne peut pas s'appliquer (puisque l'on doit diviser par 0).
        C'est d'autant plus embêtant lorsque l'on remarque que si $Y$ admet une densité, alors $\left\{ Y = y \right\}$ est de probabilité nulle pour tout $y$ \ldots.

        Cependant, si $\left\{ Y = y \right\}$ n'est pas de probabilité nulle, alors on a effectivement (mais c'est un \og coup de chance \fg) :
        \begin{displaymath}
            \forall A \in \B(\R), \Pb_{X | Y = y}(A) = \dfrac{\Pb(X \in A, Y = y )}{\Pb(Y = y)}
        \end{displaymath}
    \end{rem}

    Ces nouvelles lois conditionnelles vont nous permettre d'énoncer un théorème qui ressemble fortement à celui de Fubini.

    \begin{thm}[\textbf{Théorème de Fubini conditionnel}]
        Soit $X$ et $Y$ deux variables aléatoires réelles, et $g : \R \times \R \rightarrow \R$ une fonction borélienne, telle que la variable aléatoire réelle $g \circ (X,Y)$ admette une espérance.
        Alors :
        \begin{displaymath}
            \esp(g \circ (X,Y)) = \int_{\R} \int_{\R} g(x,y) \textnormal{d} \Pb_{X,Y} (x,y) = \int_{\R} \left( \int_\R g(x,y) \textnormal{d}\Pb_{X | Y=y} (x) \right) \textnormal{d} \Pb_Y (y)
        \end{displaymath}
    \end{thm}

    Enonçons-le maintenant de même pour des vecteurs aléatoires :

    \begin{thm}[\textbf{Théorème de Fubini conditionnel}]
        Soit $X$ et $Y$ deux vecteurs aléatoires à valeurs dans $\R^m$ et $\R^n$, et $g : \R^m \times \R^n \rightarrow \R$ une fonction borélienne, telle que la variable aléatoire réelle $g \circ (X,Y)$ admette une espérance.
        Alors :
        \begin{displaymath}
            \esp(g \circ (X,Y)) = \int_{\R^m} \int_{\R^n} g(x,y) \textnormal{d} \Pb_{X,Y} (x,y) = \int_{\R^n} \left( \int_{\R^m} g(x,y) \textnormal{d}\Pb_{X | Y=y} (x) \right) \textnormal{d} \Pb_Y (y)
        \end{displaymath}
    \end{thm}

    \begin{rem}
        Ainsi, dans le cas où $X$ et $Y$ ne sont pas indépendantes, ce théorème permettra de faire facilement les calculs.
        Si elles le sont, alors par définition de l'indépendance de deux variables aléatoires, le théorème de Fubini \og original \fg{} s'applique directement (voir le théorème \ref{thm:independance_esperance}).
    \end{rem}

    Enonçons maintenant un résultat que l'on attend intuitivement pour les lois conditionnelles : que se passe-t-il lorsque l'on cherche la loi conditionnelle d'une variable aléatoire, variable dépendant explicitement de celle par laquelle on conditionne ?

    \begin{thm}
        Soit $X$ et $Y$ deux vecteurs aléatoires à valeurs dans $\R^m$ et $\R^n$, et $g : \R^m \times \R^n \rightarrow \R$ une fonction borélienne.
        Alors pour tout $y \in \R^n$ :
        \begin{displaymath}
            \forall A \in \B(\R), \Pb_{g(X, Y) | Y = y}(A) = \Pb_{g(X, y) | Y = y}(A)
        \end{displaymath}
    \end{thm}

    Par ailleurs, voici une nouvelle caractérisation d'indépendance pour deux variables aléatoires réelles (ou deux vecteurs aléatoires) :

    \begin{thm}(\textbf{Caractérisation de l'indépendance de deux variables aléatoires à l'aide des lois conditionnelles})
        Soit $X$ et $Y$ deux variables aléatoires réelles (ou deux vecteurs aléatoires à valeurs dans $\R^m$ et $\R^n$).
        $X$ et $Y$ sont indépendantes ssi les $\Pb_{X | Y = y}$ sont indépendantes de $y$ ssi $\Pb_{X | Y = y} = \Pb_X$.
    \end{thm}

    \subsection{Espérance conditionnelle}

    Cette nouvelle section n'a \textit{a priori} rien à voir avec la section précédente.
    En réalité, les notions développées ci-dessus rejoignent celles ci-dessous en fin de section.

    Dans toute cette section, on se donne :
    \begin{itemize}
        \itemb un espace probabilisé $\left( \Omega, \A, \Pb \right)$ ;
        \itemb une variable aléatoire réelle $X$ ;
        \itemb une sous-tribu $\B$ de $\A$.
    \end{itemize}

    Comme fonction $\A$-mesurable, $X$ n'est pas nécessairement $\B$-mesurable.
    Cependant, il est commode \og d'approximer \fg{} $X$ à l'aide d'une fonction $B$-mesurable : c'est l'objet de la proposition suivante, qui effectue cette \og approximation \fg{} en utilisant un critère basé sur l'espérance.

    \begin{prop}
        Il existe une fonction $\tilde{X}$ $\B$-mesurable telle que pour tout $B \in \B$ :
        \begin{displaymath}
            \esp\left( \mathds{1}_B X \right) = \esp\left( \mathds{1}_B \tilde{X} \right)
        \end{displaymath}

        Cette fonction est unique à égalité $\Pb$-presque-partout près (\textit{ie} deux fonctions qui vérifient cette propriété diffèrent au plus sur un ensemble négligeable).
    \end{prop}

    \begin{rem}
        La fonction $\tilde{X}$ est donc une variable aléatoire réelle sur l'espace probabilité $\left( \Omega, \B, \Pb \right)$ (et évidemment, \textit{a fortiori} une variable aléatoire sur $\left( \Omega, \A, \Pb \right)$).
    \end{rem}

    La variable aléatoire $\tilde{X}$ est justement ce que l'on appelle \og espérance conditionnelle \fg.
    La définition suivante synthétise donc la proposition précédente :

    \begin{defi}
        \textbf{L'espérance de $X$ conditionnée par $\B$ $\esp(X | \B)$} est une variable aléatoire sur $\left( \Omega, \B, \Pb \right)$ telle que :
        \begin{displaymath}
            \forall B \in \B, \esp\left( \mathds{1}_B X \right) = \esp\left( \mathds{1}_B \esp(X | \B) \right)
        \end{displaymath}

        Elle existe toujours, et est unique à égalité $\Pb$-presque-partout près.
    \end{defi}

    \begin{rem}
        Attention, \textbf{l'espérance d'une variable aléatoire réelle conditionnée par une tribu est une nouvelle variable aléatoire !}
    \end{rem}

    Désormais, on peut introduire la notion d'espérance conditionnelle conditionnée par une variable aléatoire.
    On utilise pour cela la notion de tribu engendrée par une variable aléatoire (cf.\ définition~\ref{def:tribu_engendree})

    \begin{defi}
        Soit $Y$ une variable aléatoire quelconque.
        \textbf{L'espérance de $X$ conditionnée par la variable aléatoire $Y$ $\esp(X | Y)$} est une variable aléatoire sur $\left( \Omega, \sigma(Y), \Pb \right)$, définie par :
        \begin{displaymath}
            \esp(X | Y) \coloneqq \esp(X | \sigma(Y))
        \end{displaymath}
    \end{defi}

    \begin{rem}
        $X$ doit être une variable aléatoire \textbf{réelle}, puisque l'on souhaite calculer son espérance.
        Cependant, $Y$ peut être quelconque (on ne fait que conditionner par celle-ci).
    \end{rem}

    \begin{prop}
        Soit $Y$ une variable aléatoire quelconque, à valeurs dans un espace topologique $E$.
        Il existe $\psi : E \rightarrow \R$ borélienne, telle que $\esp(X | Y) = \psi(Y)$ : la nouvelle variable aléatoire $\esp(X | Y)$ est une fonction de $Y$.
    \end{prop}

    \begin{rem}[\textbf{Généralisation aux vecteurs aléatoires}]
        L'espérance conditionnée par une sous-tribu et l'espérance conditionnée par une variable aléatoire peuvent se généraliser si $X$ n'est plus une variable aléatoire réelle mais un vecteur aléatoire.
        Il suffit simplement de faire de même que pour la définition~\ref{def:vecteur_esperance}.
    \end{rem}

    Faisons maintenant le lien avec la section précédente.
    \textit{A priori}, les lois conditionnelles n'ont rien à voir avec les espérances conditionnelles.
    Cependant, comme les nouvelles lois conditionnelles sont bien des lois de probabilité, on peut définir une nouvelle espérance en les utilisant :

    \begin{defi}
        Soit $X$ et $Y$ deux variables aléatoires réelles (resp.\ deux vecteurs aléatoires à valeurs dans $\R^m$ et $\R^n$ respectivement) et $y \in \R$ (resp.\ $y \in \R^n$).
        \textbf{L'espérance conditionnelle de $X$ sachant $Y = y$} existe toujours et est la quantité réelle (resp.\ vectorielle):
        \begin{displaymath}
            \esp(X | Y = y) \coloneqq \int_{\R} x \textnormal{d} \Pb_{X | Y=y}(x)
        \end{displaymath}
    \end{defi}

    \begin{rem}
        Pour définir cette espérance conditionnelle, il ne suffit pas que $X$ soit une variable aléatoire réelle (ou un vecteur aléatoire).
        Il est en effet nécessaire que $Y$ soit également une variable aléatoire réelle (ou un vecteur aléatoire), car le théorème d'existence des lois conditionnelles n'a été énoncé que dans ce cas.
    \end{rem}

    \begin{rem}
        Attention à ne pas confondre l'espérance conditionnelle de $X$ sachant $Y = y$ et l'espérance conditionnelle de $X$ sachant $Y$.
        La première est une quantité réelle (ou vectorielle), alors que la deuxième est une variable aléatoire réelle (ou un vecteur aléatoire).

        On peut cependant faire le lien entre les deux, et c'est l'objet de ce qui suit.
    \end{rem}

    \begin{thm}
        Soit $X$ et $Y$ deux variables aléatoires réelles (resp.\ deux vecteurs aléatoires).
        Une espérance de $X$ conditionnée par $Y$ est donnée par :
        \begin{displaymath}
            \omega \longmapsto \esp(X | Y = Y(\omega))
        \end{displaymath}
        qui est donc une nouvelle variable aléatoire.

        Autrement dit, si $N \subset \Omega$ est négligeable, on a :
        \begin{displaymath}
            \forall \omega \in \Omega \setminus N, \esp(X | Y)(\omega) = \esp(X | Y = Y(\omega))
        \end{displaymath}
    \end{thm}

    \subsection{Lois et espérance conditionnelles dans deux cas particuliers}

    Maintenant que nous avons établi la théorie qui encadre les lois conditionnelles et les espérances conditionnelles, nous l'énonçons dans deux cas particuliers.
    Pour cela, on se donne un espace probabilisé $\left( \Omega, \A, \Pb \right)$.

    Enonçons le résultat dans le cas où le couple $\left( X, Y \right)$ admet une densité.

    \begin{thm}
        Soit $X$ et $Y$ deux variables aléatoires (resp.\ deux vecteurs aléatoires à valeurs dans $\R^m$ et $\R^n$).
        Supposons que la loi jointe $\left( X, Y \right)$ admette une densité $f_{\left( X, Y \right)}: \R^2 \rightarrow \R$ (resp.\ $f_{\left( X, Y \right)}: \R^{m+n} \rightarrow \R$).

        Soit $y \in \R$ (resp.\ $y \in \R^n$) tel que $f_Y(y) > 0$.

        Alors la loi conditionnelle sur $\left( \R, \B(R) \right)$ (resp.\ sur $\left( \R^m, \B(R^m) \right)$) $\Pb_{X | Y = y}$, qui existe toujours, admet une densité, notée $f_{X | Y=y} : \R \rightarrow \R$ (resp.\ $f_{X | Y=y} : \R^m \rightarrow \R$) qui vérifie pour tout $x \in \R$ (resp.\ pour tout $x \in \R^m$) :
        \begin{displaymath}
            f_{X | Y = y}(x) = \dfrac{f_{\left( X, Y \right)}(x,y)}{f_Y(y)}
        \end{displaymath}
    \end{thm}

    Maintenant, énonçons-le dans le cas où $Y$ est une variable aléatoire discrète.
    Il s'agit simplement d'un énoncé plus général de la remarque~\ref{rem:warning_cond} : il y a coïncidence entre la loi conditionnelle et la probabilité conditionnée par un évènement.
    Comme d'habitude, on se place sans perte de généralité dans le cas où $Y$ est à valeurs dans $\N$.

    \begin{thm}
        Soit $X$ une variable aléatoire réelle (resp.\ un vecteur aléatoire à valeurs dans $\R^m$), et $Y$ une variable aléatoire discrète.

        Soit $y \in \N$ tel que $\Pb(Y = y) > 0$.

        Alors la loi conditionnelle sur $\left( \R, \B(R) \right)$ (resp.\ sur $\left( \R^m, \B(R^m) \right)$) $\Pb_{X | Y = y}$, qui existe toujours, vérifie, pour tout $A \in \B(\R)$ (resp.\ pour tout $A \in \B(\R^m)$) :
        \begin{displaymath}
            \Pb_{X | Y = y}(A) = \Pb(X \in A | Y = y)
        \end{displaymath}
    \end{thm}

    \subsection{Deux théorèmes importants}

    On se donne un espace probabilisé $\left( \Omega, \A, \Pb \right)$.

    Un premier théorème important est le pendant de la \textbf{formule des probabilités totales}, mais pour les lois conditionnelles.

    \begin{thm}[\textbf{Formule du balayage conditionnel}]
        Soit $X$ et $Y$ deux variables aléatoires réelles (resp.\ deux vecteurs aléatoires à valeurs dans $\R^m$ et $\R^n$).
        Pour tout $A \in \B(\R)$ (resp.\ pour tout $A \in \B(\R^m)$), on a :
        \begin{displaymath}
            \Pb_X(A) = \int_{\R} \Pb_{X | Y = y} (A) \textnormal{d} \Pb_Y(y)
        \end{displaymath}
        et resp.\ :
        \begin{displaymath}
            \Pb_X(A) = \int_{\R^n} \Pb_{X | Y = y} (A) \textnormal{d} \Pb_Y(y)
        \end{displaymath}
    \end{thm}

    \begin{rem}
        Si $Y$ est discrète, d'après les résultats de la section précédente, on retrouve la formule des probabilités totales (les $\left\{ Y = y \right\}_{y \in \N}$ partitionnent $\Omega$) :
        \begin{displaymath}
            \Pb_X(A) = \sum_{y = 0}^{+ \infty} \Pb(X \in A | Y = y) \Pb(Y = y)
        \end{displaymath}
    \end{rem}

\end{document}