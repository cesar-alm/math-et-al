\documentclass[../integ-proba.tex]{subfiles}

\begin{document}

Maintenant que nous connaissons les notions de base concernant la théorie de la mesure, nous sommes en mesure de définir l'intégrale de Lebesgue.
Cette intégrale est en quelque sorte une généralisation de l'intégrale de Riemann.

\section{Pourquoi définir une nouvelle intégrale ?}

Deux raisons principales peuvent nous amener à définir une nouvelle intégrale.

D'une part, l'intégrale de Riemann ne permet de calculer l'intégrale que de fonctions \textbf{continues par morceaux}.
Or, cela est très restrictif : par exemple, prenons la fonction $\mathds{1}_{\mathbb{Q}}$.
Elle n'est pas continue par morceaux, donc n'est pas intégrable au sens de Riemann.
Pourtant, pour la mesure de Lebesgue, $\mathbb{Q}$ est négligeable (voir l'exemple \ref{ex:ratneg}).
Ainsi, $\mathds{1}_{\mathbb{Q}}$ est Lebesgue-presque-partout égale à $0$.
On aurait donc envie de dire que \og l'influence de $\mathbb{Q}$ est si petite, qu'elle n'a aucun impact dans l'intégrale \fg.
En disant cela, on veut en fait dire que l'intégrale de $\mathds{1}_{\mathbb{Q}}$ serait égale à l'intégrale de la fonction nulle, qui vaut zéro.
Ce faisant, \textbf{on pourrait donner un sens à l'intégrale d'une fonction qui n'est pas continue par morceaux !}

Plus généralement, l'intégrale de Lebesgue permettra de définir l'intégrale d'une fonction mesurable, et non plus d'une fonction continue par morceaux.

D'autre part, l'intégrale de Riemann utilise la notion de longueur :
par exemple, si $f$ est la fonction constante égale à 1, nous avons

$$
\int_0^1f(x)\text{d}x = 1 > 2 = \int_0^2f(x)\text{d}x
$$

Mais il est impossible de \og mesurer différemment \fg.
En effet, si une mesure $\mu$ vérifie $\mu\left(\left[1,2\right]\right)=0$, alors on aimerait pouvoir dire que les deux intégrales ci-dessus, calculées avec $\mu$, seraient égales.
Plus généralement, \textbf{on aimerait pouvoir intégrer dans n'importe-quel espace par n'importe-quelle mesure.}
L'intégrale de Lebesgue répondra à ce besoin.

\section{Définition de l'intégrale de Lebesgue pour des fonctions positives}
\label{sect:defintpos}
Les fonctions positives ont une place privilégiée dans la théorie de l'intégration de Lebesgue.
En effet, leurs intégrales sont toujours définies (quitte à valoir $+\infty$) :

\begin{defi}
  Soit $\left(E,\mathcal{A},\mu\right)$ un espace mesuré et $f:E\rightarrow\mathbb{R}_+$ mesurable.
  \textbf{La} $\mu $\textbf{ intégrale de Lebesgue } $\displaystyle\int_Ef(x)\text{d}\mu(x)$ existe toujours et est l'unique élément de $\mathbb{R}_+\cup\{+\infty\}$ vérifiant :
  \begin{itemize}
    \itemb \textbf{la relation mesure-intégrale} : pour tout ensemble $X$ $\mathcal{A}$-mesurable,
    $$
    \int_E\mathds{1}_X(x)\text{d}\mu(x)=\mu(X)
    $$ 
    \itemb \textbf{l'hypothèse de linéarité} : pour toute fonction mesurable $g:E\rightarrow\mathbb{R}_+$ et pour tous $\left(\alpha,\beta\right) \in \left(\mathbb{R}_+^*\right)^2$,
    $$
    \int_E (\alpha f + \beta g)(x) \text{d}\mu(x)=\alpha\int_Ef(x)\text{d}\mu(x) + \beta\int_Eg(x)\text{d}\mu(x)
    $$
    \itemb \textbf{l'hypothèse de convergence monotone} : pour toute suite croissante de fonctions mesurables positives $f_n:E\rightarrow\mathbb{R}_+$ convergeant simplement vers $f$,
    $$
    \lim_{n\rightarrow+\infty}\int_Ef_n(x)\text{d}\mu(x)=\int_Ef(x)\text{d}\mu(x)
    $$
  \end{itemize}
\end{defi}

\begin{rem}
  L'intégrale d'une fonction mesurable positive vérifie immédiatement, par définition, ces trois hypothèses fondamentales, qui seront centrales dans la suite de ce document.
  Attention cependant : les deux hypothèses de mesurabilité et de positivité sont absolument nécessaires.
\end{rem}

\begin{defi}
  Soit $\left(E,\mathcal{A},\mu\right)$ un espace mesuré et $f:E\rightarrow\mathbb{R}_+$ mesurable.
  $f$ est dite $\mu$\textbf{-intégrable} lorsque $\displaystyle\int_Ef(x)\text{d}\mu(x) < +\infty$.
\end{defi}

\begin{rem}
  \label{rem:attint}
  On fera donc bien attention au sens précis du mot \og intégrable \fg.
  Il est plus fort de dire qu'une fonction est intégrable, que de dire que l'intégrale d'une fonction existe.
  Par exemple, une fonction mesurable positive admet toujours une intégrale, mais elle n'est pas forcément intégrable.
\end{rem}

\begin{defi}
  Soit $f:E\longrightarrow\mathbb{R}$, et $F\subset E$.
  On dit que $f$ est \textbf{intégrable sur le sous ensemble} $F$ lorsque $f\mathds{1}_F$ est intégrable.
\end{defi}

\begin{rem}
  Pour avoir une chance que $f$ soit intégrable sur $F$, il est nécessaire que $f\mathds{1}_F$ soit mesurable.
\end{rem}

\section{Propriétés de l'intégrale positive}

% TODO changement de mesures

\section{Définition de l'intégrale de Lebesgue pour des fonctions signées}

Commençons par des résultats préliminaires qui seront utiles par la suite :

\begin{defi}
  \label{def:partieposneg}
  Soit $E$ un ensemble.
  Soit $f:E\longrightarrow\mathbb{R}$.
  Les \textbf{partie positive} et \textbf{partie négative} de $f$, respectivement $f_+$ et $f_-$, sont définies comme suit :
  $$
  \left\{
    \begin{array}{rcl}
      f_+&:=&\max(0,f)\\
      f_-&:=&\min(0,f)
    \end{array}
  \right.
  $$
\end{defi}

\begin{prop}
  Soit E un ensemble et $f:E\longrightarrow\mathbb{R}$.
  Alors :
  $$
  f = f_+ - f_-
  $$
\end{prop}

Le résultat suivant est une conséquence directe de la proposition $\ref{prop:compmes}$ :

\begin{prop}
  \label{prop:partiemes}
  Soit $\left(E,\mathcal{A}\right)$ un espace mesurable et $f:E\longrightarrow\mathbb{R}$ une fonction mesurable.
  Alors $f^+$ et $f_-$ sont mesurables
\end{prop}

Désormais, passons à la définition de l'intégrale d'une fonction signée. Celle-ci arrive volontairement dans une nouvelle section, pour souligner la différence avec l'intégrale d'une fonction positive.
En effet, comme cela a été souligné au début de la section \ref{sect:defintpos}, les fonctions positives occupent une place privilégiée dans la théorie de l'intégration de Lebesgue : elles vérifient des résultats qui ne sont pas sytématiquement vrais pour les fonctions signées.

La définition de l'intégrale pour une fonction signée est un premier exemple soulignant cette différence :

\begin{defi}
  Soit $\left(E,\mathcal{A},\mu\right)$ un espace mesuré et $f:E\rightarrow\mathbb{R}$ mesurable.
  \textbf{La} $\mu$ \textbf{intégrale de Lebesgue} $\displaystyle\int_E f(x) \text{d}\mu(x)$ existe dès que $f_+$ ou $f_-$ est $\mu$-intégrable.
  Dans ce cas, on a :
  $$
  \int_E f(x) \text{d}\mu(x)=\int_Ef_+\text{d}\mu(x) - \int_Ef_-(x)\text{d}\mu(x)
  $$
\end{defi}

\begin{rem}
  Contrairement au cas positif, \textbf{l'intégrale d'une fonction signée n'existe pas systématiquement !}
\end{rem}

\begin{defi}
  Soit $\left(E,\mathcal{A},\mu\right)$ un espace mesuré et $f:E\rightarrow\mathbb{R}$ mesurable.
  $f$ est dite $\mu$\textbf{-intégrable} lorsque $f_+$ et $f_-$ sont $\mu$-intégrables.
\end{defi}

\begin{rem}
  On notera que, de manière équivalente, \textbf{une fonction signée est intégrable ssi l'intégrale de} $f$ \textbf{existe, et est finie.}
\end{rem}

\begin{rem}
  Comme dans le cas positif (remarque \ref{rem:attint}), le mot \og intégrable\fg a un sens bien précis !
\end{rem}

\begin{thm}
  Soit $\left(E, \mathcal{A}, \mu\right)$ un espace mesuré et $f:E\longrightarrow\mathbb{R}$ mesurable. $f$ est $\mu$-intégrable ssi $\left|f\right|$ est $\mu$-intégrable.
  On dit que \textbf{l'intégrale de Lebesgue est absolue}.
\end{thm}

\begin{rem}
  Pour vérifier qu'une fonction signée est intégrable, on vérifiera en pratique que sa valeur absolue est intégrable.
\end{rem}

Le théorème suivant assure que cette nouvelle intégrale vérifie bien les besoins que l'on avait exprimé en début de chapitre.

\begin{thm}
  L'intégrale de Lebesgue prolonge l'intégrale de Riemann.
\end{thm}

\section{Démontrer en pratique des résultats sur les intégrales}

Pour montrer des résultats sur des fonctions mesurables positives, on utilise la décomposition vue au théorème \ref{thm:decompmes}.
\begin{itemize}
  \itemb On montre le résultat sur des fonctions indicatrices mesurables.
  \itemb Puis on le montre sur une somme de fonctions indicatrices mesurables. On a donc montré le résultat pour les fonctions étagées mesurables.
  \itemb En vertu de l'hypothèse de convergence monotone de l'intégrale et de la décomposition fournie par le théorème \ref{thm:decompmes}, on passe à la limite.
  Le résultat est alors démontré pour n'importe-quelle fonction mesurable positive.
\end{itemize}

En général, les résultats pour les fonctions signées se déduiront des résultats pour les fonctions positives de la manière suivante :
\begin{itemize}
  \itemb On décompose une fonction signée mesurable $f:E\longrightarrow\mathbb{R}$ en sa partie positive $f^+$ et sa partie négative $f_-$ (voir la définition \ref{def:partieposneg}).
  \itemb Ces deux fonctions étant positives (et encore mesurables par la proposition \ref{prop:partiemes}), on applique leur applique un résultat vrai pour les fonctions positives.
  \itemb On utilise l'identité $f=f^++f_-$ pour obtenir le résultat final.
\end{itemize}

On rappelle toutefois qu'un certain nombre de résultats vrais pour les fonctions positives ne le sont pas pour les fonctions signées.

%TODO Exemple

\section{Généralisation aux fonctions à valeurs dans $\mathbb{R}^m$}

Nos fonctions étaient jusqu'ici à valeurs dans $\mathbb{R}$. Qu'en est-il si elle sont à valeurs dans $\mathbb{R}^m$ ?

Soit $f:E\longrightarrow \mathbb{R}^m$. Décomposons $f$ :

$$
f = \sum_{i=1}^{m}f_ie_i
$$

où :
\begin{itemize}
  \itemb $\left(f_i\right)_{i\in[\![1,m]\!]}$ sont les \textbf{composantes} de $f$.
  \itemb $\left(e_i\right)_{i\in[\![1,m]\!]}$ est la base canonique de $\mathbb{R}^m$.
\end{itemize}

Posons alors les définitions suivantes :

\begin{defi}
  L'intégrale de $f$ existe lorsque les $m$ intégrales des $f_i$ existent.
  Lorsqu'elle existe, l'intégrale de $f$ vaut :
  $$
  \int_Ef = \sum_{i=1}^{m}\int_Ef_i
  $$
  (on peut permuter vecteurs et intégrale)
\end{defi}

\begin{defi}
  $f$ est dite \textbf{intégrable} lorsque les $m$ fonctions $f_i$ sont intégrables.
\end{defi}

Avec ces deux définitions, nous avons très facilement généralisé la notion d'intégrale lorsque les fonctions sont à valeurs dans $\mathbb{R}^m$.
Tous les résultats d'intégrations valables pour des fonctions à valeurs dans $\mathbb{R}$ pourront ainsi se transposer très facilement dans le cas général.

\end{document}