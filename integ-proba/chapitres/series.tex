\documentclass[../integ-proba.tex]{subfiles}

\begin{document}

Dans ce chapitre, nous allons voir les principaux résultats d'intégration, lorsque l'on muni l'intégrale de Lebesgue de la mesure de comptage.
Nous allons voir que cela nous permet d'explorer la théorie des séries.

Dans tout ce chapitre, on considère l'espace mesuré $\left(\mathbb{N}^n, \mathcal{P}(\mathbb{N}^n), c\right)$, où $c$ est la mesure de comptage.
Nos fonctions prendront donc leurs valeurs dans $\mathbb{N}^n$.
Dans la pratique, on prendra souvent $n=1$.

\section{Lien avec les séries}

\section{Que deviennent les résultats généraux dans ce cas particulier ?}

\end{document}