\documentclass[../integ-proba.tex]{subfiles}

\begin{document}

    \chapter{Généralités sur les probabilités}

    \section{Premières définitions}

    Commençons par définir les notions fondamentales de probabilités, à l'aide du formalisme introduit dans la partie précédente.

    \begin{defi}
        On appelle \textbf{espace probabilisable} un espace mesurable, dans le contexte des probabilités.
    \end{defi}

    \begin{defi}
        Soit $\left(\Omega, \mathcal{A}\right)$ un espace probabilisable.
        Une \textbf{mesure de probabilité} (ou \textbf{loi de probabilité}) $\mathbb{P}$ sur $\left(\Omega, \mathcal{A}\right)$ est une mesure sur $\left(\Omega, \mathcal{A}\right)$, à valeurs dans $\left[0,1\right]$, telle que $\mathbb{P}\left(\Omega\right)=1$.
    \end{defi}

    \begin{defi}
        On appelle \textbf{espace probabilisé} un espace probabilisable muni d'une mesure de probabilité.
    \end{defi}

    \section{Deux cas particuliers de mesures de probabilités}

    Il existe un cas particulier où les mesures de probabilité sont simples à exprimer : celui où elles sont définies par une intégrale (cf.\ section~\ref{sec:mes_defi_int}.
    Il permet de grandement simplifier les calculs.

    Ce cas particulier se décline en deux, selon si la mesure considérée est celle de Lebesgue ou celle de comptage.

        \subsection{Mesures de probabilité à densité}
            \begin{defi}
                On se place sur $\mathbb{R}$ muni de la tribu borélienne et de la mesure de Lebesgue $\lambda$.
                Soit $f$ une fonction positive et intégrable d'intégrale 1.
                Soit $\mu:=f \lambda$.

                Alors $\mu$ est une mesure de probabilité, et on dit alors que $\mu$ est \textbf{à densité}.
            \end{defi}

            \begin{rem}
                Notons que la réciproque est fausse : il existe des mesures sur $\mathbb{R}$ qui n'admettent pas de densité.
            \end{rem}

            % TODO examples


        \subsection{Mesure de probabilité discrètes}
            Le deuxième cas particulier concerne les espaces discrets.
            Sans perte de généralité, on se place uniquement dans le cas de $\mathbb{N}$.

            On commence par énoncer le résultat important suivant :

            \begin{thm}
                Soit $\left(\mathbb{N}, \mathcal{P}(\mathbb{N}), \mu \right)$ un espace probabilisé.
                Il existe une unique suite $f:\mathbb{N} \rightarrow \mathbb{R}$ positive et sommable de somme 1 telle que $\mu = f c$, où $c$ est la mesure de comptage.
            \end{thm}

            \begin{rem}
                Ce résultat nous prouve que \textbf{les mesures de probabilité discrètes sont caractérisées par leurs valeurs élémentaires)} (\textit{ie} sur les singletons).

                Notons cependant que ce résultat est faux dans le cas général.
                Par exemple, dans le cas des mesures à densité, nous verrons que toutes les valeurs élémentaires sont nulles : il n'y a donc pas unicité.
            \end{rem}

            % TODO examples


\end{document}