\documentclass[../integ-proba.tex]{subfiles}

\begin{document}

\section{Premières définitions}

Commençons par définir les notions fondamentales de probabilités, à l'aide du formalisme introduit dans la partie précédente.

\begin{defi}
    On appelle \textbf{espace probabilisable} un espace mesurable, dans le contexte des probabilités.
\end{defi}

\begin{defi}
    Soit $\left(\Omega, \mathcal{A}\right)$ un espace probabilisable.
    Une \textbf{mesure de probabilité} (ou \textbf{loi de probabilité}) $\mathbb{P}$ sur $\left(\Omega, \mathcal{A}\right)$ est une mesure sur $\left(\Omega, \mathcal{A}\right)$, à valeurs dans $\left[0,1\right]$, telle que $\mathbb{P}\left(\Omega\right)=1$.
\end{defi}

\begin{defi}
    On appelle \textbf{espace probabilisé} un espace probabilisable muni d'une mesure de probabilité.
\end{defi}



\end{document}