\documentclass[../integ-proba.tex]{subfiles}

\begin{document}

    \chapter{Convergence des suites de variables aléatoires}

    Interessons-nous désormais à la convergence des suites de variables aléatoires.
    Il n'existe pas une seule notion de convergence pour les variables aléatoires.
    Certains modes de convergence sont plus forts que d'autres, et il convient de ne pas les confondre.

    \section{Les différents modes de convergence}

    \subsection{La convergence en loi}

    Dans cette section, on se donne une suite de variables aléatoires $\left(X_n\right)_{n\in\mathbb{N}}$ quelconques à valeurs dans $E$.
    On se donne également une autre variable aléatoire $X$ à valeurs dans $E$.

    \begin{defi}
        On dit que $\left(X_n\right)_{n\in\mathbb{N}}$ \textbf{converge en loi} vers $X$, et on note $X_n \xrightarrow[]{\mathcal{L}} X$ lorsque pour toute fontion $f:E \rightarrow \mathbb{R}$ continue et bornée,
        \begin{displaymath}
            \lim_{n \to +\infty} \mathbb{E}\left(f \circ X_n\right) = \mathbb{E}\left(f \circ X\right)
        \end{displaymath}
    \end{defi}

    \begin{rem}
        Remarquons qu'il n'est pas nécessaire d'imposer une quelconque condition sur $E$, pourvu que l'on puisse définir la continuité d'une fonction qui prend ses valeurs dans $E$ (par exemple, en le munissant d'une topologie).
    \end{rem}

    \begin{rem}
        Pour ce mode de convergence, il n'est pas nécessaire que les $\left(X_n\right)_{n\in\mathbb{N}}$ et $X$ soient définies sur le même espace probabilisé.
        En effet, la définition a encore du sens même si :
        \begin{itemize}
        \itemb les $\left(X_n\right)_{n\in\mathbb{N}}$ sont définies sur une suite d'espace probabilisés $\left(\Omega_n, \mathcal{A}_n, \mathbb{P}_n\right)_{n \in \mathbb{N}}$ ;
        \itemb $X$ est défini sur un espace probabilisé $\left(\Omega, \mathcal{A}, \mathbb{P}\right)$
        \end{itemize}

        En réalité, et comme le suggère la propriété suivante, la convergence en loi ne concerne que la convergence des lois $\mathbb{P}_{X_n}$ vers $\mathbb{P}_X$.
    \end{rem}

    Pour cette proposition uniquement, on considère que les $\left(X_n\right)_{n\in\mathbb{N}}$ et que $X$ sont des variables aléatoires réelles ou discrètes (de manière à ce que la fonction de répartition ait un sens).

    \begin{prop}
        $\left(X_n\right)_{n\in\mathbb{N}}$ converge en loi vers $X$ ssi pour tout $x \in \mathbb{R}$ où la fonction de répartition $F_X$ est continue,
        \begin{displaymath}
            \lim_{n \to +\infty} F_{X_n}(x) = F_X(x)
        \end{displaymath}
    \end{prop}

    \begin{rem}
        Comme nous le verrons par la suite, ce mode de convergence est le plus faible de tous.
    \end{rem}

    \begin{rem}
        Enfin, ce mode de convergence sera celui impliqué dans le théorème~\ref{thm:tcl} (théorème centrale limite).
    \end{rem}

    \subsection{La convergence en probabilité}

    Dans cette section, on se donne une suite de variables aléatoires $\left(X_n\right)_{n\in\mathbb{N}}$ réelles ou discrètes.
    On se donne également une autre variable aléatoire $X$ réelle ou discrète.
    On suppose que toutes ces variables aléatoires sont définies sur le même espace probabilisé.

    \begin{defi}
        On dit que $\left(X_n\right)_{n\in\mathbb{N}}$ \textbf{converge en probabilité} vers $X$, et on note $X_n \xrightarrow[]{\mathbb{P}} X$ lorsque :
        \begin{displaymath}
            \forall \delta > 0, \lim_{n \to +\infty} \mathbb{P}\left(\left|X_n - X\right| \geq \delta\right) = 0
        \end{displaymath}
    \end{defi}

    \begin{rem}
        Remarquons qu'il est nécessaire que les variables aléatoires soient réelles ou discrètes : on doit en effet pouvoir utiliser la relation d'ordre habituelle sur $\mathbb{R}$.
    \end{rem}

    \begin{rem}
        Pour ce mode de convergence, il est absolument nécessaire que les $\left(X_n\right)_{n\in\mathbb{N}}$ et $X$ soient définies sur le même espace probabilisé.
        En effet, soit $n \in \mathbb{N}$ : nous sommes alors amenés à considérer la variable aléatoire $X_n - X$.
        Pour que cette fonction ait du sens, il faut en particulier que l'espace de départ soit le même.
        De plus, on doit utiliser la loi de $\left|X_n - X\right|$ : pour que celle-ci existe et ait du sens, il faut que la tribu et la mesure de probabilité de l'espace de départ soient identiques.
    \end{rem}

    Ce mode de convergence est plus fort que la convergence en loi :

    \begin{prop}
        Supposons que $\left(X_n\right)_{n\in\mathbb{N}}$ converge en probabilité vers $X$.
        Alors $\left(X_n\right)_{n\in\mathbb{N}}$ converge en loi vers $X$.
    \end{prop}

    \begin{rem}
        Enfin, ce mode de convergence sera celui impliqué dans le théorème~\ref{thm:loi_faible_gds_nbs} (loi faible des grands nombres)
    \end{rem}

    \subsection{La convergence en norme \texorpdfstring{$\mathcal{L}^r$}{Lr}}

    Dans cette section, on se donne une suite de variables aléatoires $\left(X_n\right)_{n\in\mathbb{N}}$ réelles ou discrètes.
    On se donne également une autre variable aléatoire $X$ réelle ou discrète.
    On suppose que toutes ces variables aléatoires sont définies sur le même espace probabilisé.

    \begin{defi}
        On dit que $\left(X_n\right)_{n\in\mathbb{N}}$ \textbf{converge en norme} $\mathcal{L}^r$ vers $X$, et on note $X_n \xrightarrow[]{\mathcal{L}^r} X$ lorsque :
        \begin{displaymath}
            \lim_{n \to +\infty}\mathcal{E}\left(\left|X_n - X\right|^r\right) = 0
        \end{displaymath}
    \end{defi}

    \begin{rem}
        Remarquons qu'il est nécessaire que les variables aléatoires soient réelles ou discrètes : on doit en effet pouvoir donner un sens au moment d'ordre $r$.
        Néanmoins, si $r=1$, on peut également étendre la définition aux vecteurs aléatoires.
    \end{rem}

    \begin{rem}
        Lorsque $r=1$, on parle de \textbf{convergence en moyenne}.
        Lorsque $r=2$, on parle de \textbf{convergence en moyenne quadratique}.
    \end{rem}

    \begin{rem}
        Pour ce mode de convergence, tout comme pour la convergence en probabilité, il est absolument nécessaire que les $\left(X_n\right)_{n\in\mathbb{N}}$ et $X$ soient définies sur le même espace probabilisé.
        La justification d'ailleurs sensiblement la même que pour la convergence en probabilité : pour calculer l'espérance, on est amenés à considérer la loi de probabilités de $X_n - X$.
    \end{rem}

    Ce mode de convergence est plus fort que la convergence en probabilité, et \textit{a fortiori} plus fort que la convergence en loi :

    \begin{prop}
        Supposons que $\left(X_n\right)_{n\in\mathbb{N}}$ converge en norme $\mathcal{L}^r$ vers $X$.
        Alors $\left(X_n\right)_{n\in\mathbb{N}}$ converge en probabilité vers $X$.
    \end{prop}

    Par ailleurs, on a également l'implication suivante :

    \begin{prop}
        Supposons que $\left(X_n\right)_{n\in\mathbb{N}}$ converge en norme $\mathcal{L}^r$ vers $X$.
        Soit $s$ tel que $1 \leq s \leq r$.
        Alors $\left(X_n\right)_{n\in\mathbb{N}}$ converge en norme $\mathcal{L}^s$ vers $X$.
    \end{prop}

    \begin{rem}
        Enfin, ce mode de convergence sera un des modes de convergence impliqué dans le théorème~\ref{thm:loi_forte_gds_nbs} (loi forte des grands nombres).
    \end{rem}

    \subsection{La convergence presque-sûre}

    Ce mode de convergence est celui qui se rapproche le plus de la converence simple.
    Il faut cependant garder à l'esprit que c'est loin d'être le seul mode de convergence existant.

    Dans cette section, on se donne une suite de variables aléatoires $\left(X_n\right)_{n\in\mathbb{N}}$ quelconques à valeurs dans $E$.
    On se donne également une autre variable aléatoire $X$ à valeurs dans $E$.
    On suppose que toutes ces variables aléatoires sont définies sur le même espace probabilisé.

    \begin{defi}
        On dit que $\left(X_n\right)_{n\in\mathbb{N}}$ \textbf{converge presque-sûrement} vers $X$, et on note $X_n \xrightarrow[]{\text{p.s.}} X$ lorsque pour $\mathbb{P}$-presque-tout $\omega \in \Omega$,
        \begin{displaymath}
            \lim_{n \to +\infty} X_n(\omega) = X(\omega)
        \end{displaymath}
    \end{defi}

    \begin{rem}
        Remarquons qu'il n'est pas nécessaire d'imposer une quelconque condition sur $E$, pourvu que l'on puisse définir la continuité d'une fonction qui est à valeurs dans $E$ (par exemple, en le munissant d'une topologie).
        Cependant, dans la plupart des cas, les variables aléatoires seront discrètes, réelles, ou seront des vecteurs aléatoires.
    \end{rem}

    \begin{rem}
        Pour ce mode de convergence, il nécessaire que les $\left(X_n\right)_{n\in\mathbb{N}}$ et $X$ soient définies sur le même espace probabilisé, puisqu'on ne considère que la loi $\mathbb{P}$ dans le \og $\mathbb{P}$-presque-partout \fg.
    \end{rem}

    Ce mode de convergence est plus fort que la convergence en probabilité, et \textit{a fortiori} plus fort que la convergence en loi :

    \begin{prop}
        Supposons que $\left(X_n\right)_{n\in\mathbb{N}}$ converge presque-sûrement vers $X$.
        Alors $\left(X_n\right)_{n\in\mathbb{N}}$ converge en probabilité vers $X$.
    \end{prop}

    \begin{rem}
        Attention, les convergences en norme $\mathcal{L}^r$ et la convergence presque-sûre ne sont pas équivalentes !
        De plus, il n'existe pas sans hypothèse supplémentaire d'implications entre ces deux modes de convergence.
    \end{rem}

    % todo qqes résultats intéressant d'implications avec des hypothèses supplémentaires.

    \begin{rem}
        Enfin, ce mode de convergence sera l'autre mode de convergence impliqué dans le théorème~\ref{thm:loi_forte_gds_nbs} (loi forte des grands nombres).
    \end{rem}

\end{document}