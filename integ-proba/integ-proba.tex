\documentclass[french]{report}

% always there : font encoding
\usepackage[T1]{fontenc}
% always there : vectorial fonts
\usepackage{lmodern}
% always there : french compatibility
\usepackage{babel}
\def\frenchtablename{Tableau}

\usepackage{dsfont}

% hyperref to access external refs; xurl so they cut properly at the end of line
\usepackage{hyperref}
\usepackage{xurl}

% math stuff : necessary
\usepackage{amsmath}
\usepackage{amssymb}
\usepackage{amsfonts}
\usepackage{amsthm}

\theoremstyle{plain}
\newtheorem{thm}{Théorème}[section]
\newtheorem{prop}{Proposition}[section]
\newtheorem{cor}{Corollaire}[section]

\theoremstyle{definition}
\newtheorem{defi}{Définition}[section]
\newtheorem{exemple}{Exemple}[section]

\theoremstyle{remark}
\newtheorem{rem}{Remarque}[section]

% large projet management
\usepackage{subfiles}

% bullets for lists
\newcommand\itemb{\item[$\bullet$]}

\begin{document}

\title{Intégration et probabilités}

\author{César Almecija
\\ étudiant en première année aux MINES ParisTech
\\ \href{mailto:cesar.almecija@mines-paristech.fr}{cesar.almecija@mines-paristech.fr}}

\maketitle

\tableofcontents

\subfile{chapitres/avant_propos}

\part{Théorie de la mesure et intégration}

\subfile{chapitres/intro_theorie_mesure}

\subfile{chapitres/construction_integrale}

\subfile{chapitres/integrale_avec_mesure_lebesgue}

\subfile{chapitres/series}

\part{Probabilités}

\subfile{chapitres/probas_generalites}

\subfile{chapitres/probas_va}

\subfile{chapitres/probas_conv}

\end{document}