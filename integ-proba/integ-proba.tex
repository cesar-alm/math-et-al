\documentclass[french]{report}
\usepackage{babel}
\usepackage{dsfont}
\usepackage{hyperref}
\usepackage{lmodern}
\usepackage{amsmath}
\usepackage{amssymb}
\usepackage{amsfonts}
\usepackage{amsthm}
\usepackage[T1]{fontenc}

\theoremstyle{plain}
\newtheorem{thm}{Théorème}[section]
\newtheorem{prop}{Proposition}[section]
\newtheorem*{compmes}{Proposition (Composition des fonctions mesurables)}

\theoremstyle{definition}
\newtheorem{defi}{Définition}[section]
\newtheorem{exemple}{Exemple}[section]

\theoremstyle{remark}
\newtheorem{rem}{Remarque}[section]

\newcommand\itemb{\item[$\bullet$]}

\begin{document}

\title{Intégration et probabilités}
\author{César Almecija
\\ étudiant en première année aux MINES ParisTech
\\ \href{mailto:cesar.almecija@mines-paristech.fr}{cesar.almecija@mines-paristech.fr}}

\maketitle

\chapter*{Avant-propos}
Ce document vise à résumer les notions essentielles d'intégration et de probabilités.
Son objectif est de retracer les cheminements de pensée qui mènent aux résultats essentiels, afin de mieux les retenir.
Ainsi, il pourra être avantageusement utilisé pour se refamiliariser avec ces notions, au cas-où elles auraient été oubliées.
Ce travail peut également servir de préambule à une étude plus approfondie du sujet, pour découvrir les fondements de ce domaine des mathématiques.

Néanmoins, ce document ne saurait se substituer à un ouvrage de référence. En effet, par soucis de clarté et de concision, il ne détaille pas toutes les démonstrations.
Celles-ci sont néanmoins nécessaires pour saisir l'entièreté des résultats énoncés ici. De cette manière, \textbf{il faut le lire comme un aide-mémoire ou un résumé, et non comme un cours}.

\bigskip

Ce travail se partage en trois grandes parties :
\begin{itemize}
  \itemb Une première partie commençant par une introduction à la théorie de la mesure, suivie de la construction rapide de l'intégrale de Lebesgue, et qui finit par s'attarder sur les résultats essentiels d'intégration
  \itemb Une deuxième partie qui applique la partie précédente à la théorie des probabilités
  \itemb Une troisième partie rapide qui explore la théorie des séries à l'aide des résultats précédents.
\end{itemize}

\bigskip

Ce document s'est grandement inspiré du cours \href{https://github.com/boisgera/CDIS}{Calcul Différentiel, Intégral et Stochastique (CDIS)} de l'Ecole des MINES ParisTech.
L'auteur recommande aux lecteurs intéressés de s'y référer régulièrement.

\part{Théorie de la mesure et intégration}

\chapter{Introduction à la théorie de la mesure}

Pour comprendre cette notion, partons d'un constat évident. On sait mesurer la taille d'un segment : la manière de mesurer cette grandeur est d'utiliser la \textit{longueur}.
Par exemple, la longueur du segment $\left[0,1\right]$ vaut $1$.
Plus généralement, $\lambda$, l'application qui à un segment $\left[a,b\right]$ associe sa longueur $b-a$, pourrait s'appeler dans le langage courant une \textit{mesure}, c'est-à-dire une fonction permettant d'obtenir la mesure (la taille) d'un objet.

Se posent alors les questions suivantes :
\begin{itemize}
  \itemb dans $\mathbb{R}$, comment mesurer n'importe-quel ensemble ?
  \itemb plus généralement, comment mesurer dans $\mathbb{R}^n$ ?
  \itemb mais est-il possible de \textit{tout} mesurer ? Ou faut-il se restreindre à une catégorie d'ensembles \textit{mesurables} ?
\end{itemize}

On commencera par répondre à cette dernière question.

\section{Ensembles mesurables}

Il n'est pas possible de tout mesurer ainsi. Il faut introduire une classe d'ensembles, appelée \textbf{tribu}, qui va représenter l'ensemble des ensembles \textbf{mesurables}.

Une tribu doit cependant garantir certaines propriétés de stabilité par opérations ensemblistes.
En effet, si deux ensembles sont mesurables, on aimerait pouvoir donner un sens à la mesure de leur intersection ou de leur union.
De même, si un ensemble est mesurable, il est légitime de demander que son complémentaire soit mesurable à son tour.
Enfin, demander que l'ensemble vide soit mesurable est raisonnable, et on verra par la suite que la mesure du vide doit valoir $0$, conformément à l'intuition.

Ces considérations étant faites, on définit alors une tribu comme suit :

\begin{defi}
  Une \textbf{tribu} (ou $\sigma\textbf{-algèbre}$) $\mathcal{A}$ d'un ensemble $E$ est une collection d'ensembles $\mathcal{A} \subset \mathcal{P}(E)$ vérifiant les hypothèses suivantes :
  \begin{enumerate}
    \item $\varnothing \in \mathcal{A}$
    \item $\mathcal{A}$ est stable par passage au complémentaire
    \item $\mathcal{A}$ est stable par union dénombrable
  \end{enumerate}
\end{defi}

\begin{defi}
  Un \textbf{espace mesurable} $\left(E, \mathcal{A}\right)$ est un ensemble $E$ muni d'une tribu $\mathcal{A}$.
\end{defi}

\begin{defi}
  Soit $\left(E, \mathcal{A}\right)$ un espace mesurable. Un élément $X\in \mathcal{A}$ est dit $\mathcal{A}$-\textbf{mesurable}.
\end{defi}

\begin{rem}
  Si le contexte est clair (\textit{ie} il n'y a qu'une seule tribu), on pourra omettre $\mathcal{A}$ et parler simplement d'ensemble mesurable
\end{rem}

\begin{defi}
  Soit $E$ un ensemble, et $B$ une collection d'ensembles de $E$.
  La \textbf{tribu engendrée par} $B$ est la plus petite tribu sur $E$ contenant $B$.
  De manière équivalente, c'est l'intersection de toutes les tribus de $E$ contenant $B$.
\end{defi}

\begin{exemple}
  \label{ex:borel}
  Soit $E$ un ensemble muni d'une topologie.
  La \textbf{tribu de Borel} est la tribu engendrée par les ouverts de $E$ (ou de manière équivalente, par les fermés de $E$).
\end{exemple}

\section{Définition de la mesure}

Abordons désormais la notion de mesure.
Un ensemble mesurable doit intuitivement admettre une mesure positive.
De plus, si deux ensembles sont disjoints, il est légitime de demander que la mesure de leur union soit la somme des mesures.

Cela nous amène à définir une mesure ainsi :

\begin{defi}
  Une \textbf{mesure} $\mu$ sur un espace mesurable $(E, \mathcal{A})$ est une application $\mu : \mathcal{A} \longrightarrow \mathbb{R}_+$ telle que :
  \begin{enumerate}
    \item $\mu(\varnothing) = 0$
    \item Pour une famille au plus dénombrable d'ensembles mesurables $\left(A_i\right)_{i \in I}$,
    $$\mu\left(\bigcup_{i \in I}A_i\right) = \sum_{i \in I} \mu\left(A_i\right)$$
  \end{enumerate}
\end{defi}

\begin{rem}
  Cette deuxième propriété porte le nom de $\sigma$\textbf{-additivité}.
\end{rem}

\begin{rem}
  Si $\mu(\varnothing) \neq 0$, le lecteur pourra vérifier que la mesure de n'importe-quel ensemble mesurable (\textit{a fortiori} du vide) vaut $+\infty$.
  On comprend alors pourquoi on impose que la mesure du vide soit nulle : si ce n'était pas le cas, mesurer n'aurait pour ainsi dire aucun intérêt.
\end{rem}

\begin{defi}
  Un \textbf{espace mesuré} $\left(E, \mathcal{A}, \mu\right)$ est un espace mesurable $\left(E, \mathcal{A}\right)$ muni d'une mesure $\mu$.
\end{defi}

\begin{defi}
  Soit $\left(E, \mathcal{A}, \mu\right)$ un espace mesuré.
  Un ensemble $X \subset E$ est dit \textbf{négligeable} si :
  $$\exists Y \in \mathcal{A}, 
  \left\{
    \begin{array}{c}
      \mu(Y)=0\\
      X \subset Y
    \end{array}
    \right.$$
\end{defi}

\begin{rem}
  \label{rem:completion}
  On remarque qu'un ensemble négligeable n'est pas forcément mesurable.
  Il est cepedant commode d'imposer qu'un ensemble négligeable soit nécessairement mesurable.
  Modifier une mesure et la tribu respective pour arriver à ce résultat porte le nom de \textbf{complétion d'une mesure}.

  Ce processus ne sera pas détaillé ici, mais le lecteur peut néanmois retenir qu'une mesure complétée vérifie l'équivalence suivante : \textbf{un ensemble est négligeable ssi il est mesurable, de mesure nulle}
\end{rem}

\section{Exemples importants}
Il existe trois mesures importantes à connaître.

\begin{defi}
  La \textbf{mesure de Lebesgue} est l'unique mesure sur $\mathbb{R}^n$ qui prolonge la notion de volume.
  Habituellement, sa tribu de définition est la \textbf{tribu de Lebesgue}.
\end{defi}

\begin{rem}
  On peut définir la mesure de Lebesgue sur d'autres tribus, notamment la tribu de Borel (voir \ref{ex:borel}). Mais ce faisant, la mesure ne serait pas complète (voir \ref{rem:completion})
  La tribu de Lebesgue est en fait la plus petite tribu de $\mathbb{R}^n$ permettant à cette mesure d'être complète.
  
  On ne retiendra pas plus de détails sur cette tribu, mais il est bon de retenir qu'\textbf{un élément Borel-mesurable est Lebesgue-mesurable} (la réciproque est fausse).
\end{rem}

\begin{rem}
  On remarque immédiatement que :
  \begin{itemize}
    \itemb lorsque $n=1$, cette mesure prolonge la notion de longueur vue en introduction du chapitre.
    \itemb lorsque $n=2$, cette mesure prolonge la notion de surface.
    \itemb lorsque $n=3$, cette mesure prolonge la notion de volume.
  \end{itemize}
\end{rem}


\section{Fonctions mesurables}

Lorsque deux espaces mesurables sont définis, il peut être utile de définir une classe de fonctions qui font bon ménage avec les tribus des espaces mis en jeu.

\begin{defi}
  Soit $\left(E, \mathcal{A}\right)$ et $\left(F, \mathcal{B}\right)$ deux espaces mesurables.
  Une \textbf{fonction } $\mathcal{A}$/$\mathcal{B}$\textbf{-mesurable} est une fonction $f : E \longrightarrow F$ telle que :
  $$ \forall Y \in \mathcal{B}, f^{-1}(Y) \in \mathcal{A} $$
\end{defi}

\begin{rem}
  Cette caractérisation des fonctions mesurables porte le nom de \textbf{critère de l'image réciproque}.
\end{rem}

\begin{defi}
  \label{def:fctmes2}
  Soit $\left(E, \mathcal{A}\right)$ un espace mesurable. Une fonction $\mathcal{A}$\textbf{-mesurable} est une fonction $f:E \longrightarrow \mathbb{R}^n$ telle que :
  $$\forall Y \in \mathcal{B}, f^{-1}(Y) \in \mathcal{A}$$
  où $\mathcal{B}$ est la tribu de Lebesgue sur $\mathbb{R}^n$.
\end{defi}

\begin{rem}
  De cette manière, dans le cas où une fonction est à valeur dans l'espace mesurable $\mathbb{R}^n$ muni de la tribu de Lebesgue, on omettra la mention de la tribu de l'espace d'arrivée.
  Cela permet d'alléger les énoncés, étant donné que, dans l'immense majorité des cas, les fonctions seront effectivement à valeurs dans $\mathbb{R}^n$ muni de la tribu de Lebesgue.
\end{rem}

\begin{rem}
  Dans le même esprit de simplification, si :
  \begin{itemize}
    \itemb l'espace d'arrivée est $\mathbb{R}^n$, muni de la mesure de Lebesgue
    \itemb à part la tribu de Lebesgue, il n'y a qu'une autre tribu en jeu
  \end{itemize}

  alors on dira simplement qu'une fonction vérifiant la définition \ref{def:fctmes2} est \textbf{mesurable}.
\end{rem}

Le résultat suivant peut sembler anodin mais sera important dans la partie concernant les probabilités

\begin{compmes}
  Soit $\left(E,\mathcal{A}\right)$, $\left(F,\mathcal{B}\right)$ et $\left(G,\mathcal{C}\right)$ trois espaces mesurables.
  Soit $f : E \longrightarrow F$ une fonction $\mathcal{A}$/$\mathcal{B}$-mesurable et $g : F \longrightarrow G$ une fonction $\mathcal{B}$/$\mathcal{C}$-mesurable.
  Alors $g \circ f : E \longrightarrow G$ est une fonction $\mathcal{A}$/$\mathcal{C}$-mesurable
\end{compmes}

\tableofcontents

\end{document}