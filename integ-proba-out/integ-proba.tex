\documentclass[french]{report}
\usepackage{babel}
\usepackage{dsfont}
\usepackage{hyperref}
\usepackage{lmodern}
\usepackage{amsmath}
\usepackage{amssymb}
\usepackage{amsfonts}
\usepackage{amsthm}
\usepackage[T1]{fontenc}

\usepackage{subfiles}

\theoremstyle{plain}
\newtheorem{thm}{Théorème}[section]
\newtheorem{prop}{Proposition}[section]
\newtheorem{cor}{Corollaire}[section]

\theoremstyle{definition}
\newtheorem{defi}{Définition}[section]
\newtheorem{exemple}{Exemple}[section]

\theoremstyle{remark}
\newtheorem{rem}{Remarque}[section]

\newcommand\itemb{\item[$\bullet$]}

\begin{document}

\title{Intégration et probabilités}

\author{César Almecija
\\ étudiant en première année aux MINES ParisTech
\\ \href{mailto:cesar.almecija@mines-paristech.fr}{cesar.almecija@mines-paristech.fr}}

\maketitle

\tableofcontents

\chapter*{Avant-propos}

\subfile{chapitres/avant_propos}

\part{Théorie de la mesure et intégration}

\chapter{Introduction à la théorie de la mesure}

\subfile{chapitres/intro_theorie_mesure}

\chapter{Construction de l'intégrale de Lebesgue}

\subfile{chapitres/construction_integrale}

\chapter{L'intégrale de Lebesgue munie de la mesure de Lebesgue}

\subfile{chapitres/integrale_avec_mesure_lebesgue}

\part{Probabilités}

\chapter{Généralités sur les probabilités}

\subfile{chapitres/probas_generalites}

\end{document}