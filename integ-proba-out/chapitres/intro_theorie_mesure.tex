\documentclass[../integ-proba.tex]{subfiles}

\begin{document}

Pour comprendre cette notion, partons d'un constat évident. On sait mesurer la taille d'un segment : la manière de mesurer cette grandeur est d'utiliser la \textit{longueur}.
Par exemple, la longueur du segment $\left[0,1\right]$ vaut $1$.
Plus généralement, $\lambda$, l'application qui à un segment $\left[a,b\right]$ associe sa longueur $b-a$, pourrait s'appeler dans le langage courant une \textit{mesure}, c'est-à-dire une fonction permettant d'obtenir la mesure (la taille) d'un objet.

Se posent alors les questions suivantes :
\begin{itemize}
  \itemb dans $\mathbb{R}$, comment mesurer n'importe-quel ensemble ?
  \itemb plus généralement, comment mesurer dans $\mathbb{R}^n$ ?
  \itemb mais est-il possible de \textit{tout} mesurer ? Ou faut-il se restreindre à une catégorie d'ensembles \textit{mesurables} ?
\end{itemize}

On commencera par répondre à cette dernière question.

\section{Ensembles mesurables}

Il n'est pas possible de tout mesurer ainsi. Il faut introduire une classe d'ensembles, appelée \textbf{tribu}, qui va représenter l'ensemble des ensembles \textbf{mesurables}.

Une tribu doit cependant garantir certaines propriétés de stabilité par opérations ensemblistes.
En effet, si deux ensembles sont mesurables, on aimerait pouvoir donner un sens à la mesure de leur intersection ou de leur union.
De même, si un ensemble est mesurable, il est légitime de demander que son complémentaire soit mesurable à son tour.
Enfin, demander que l'ensemble vide soit mesurable est raisonnable, et on verra par la suite que la mesure du vide doit valoir $0$, conformément à l'intuition.

Ces considérations étant faites, on définit alors une tribu comme suit :

\begin{defi}
  Une \textbf{tribu} (ou $\sigma\textbf{-algèbre}$) $\mathcal{A}$ d'un ensemble $E$ est une collection d'ensembles $\mathcal{A} \subset \mathcal{P}(E)$ vérifiant les hypothèses suivantes :
  \begin{enumerate}
    \item $\varnothing \in \mathcal{A}$
    \item $\mathcal{A}$ est stable par passage au complémentaire
    \item $\mathcal{A}$ est stable par union dénombrable
  \end{enumerate}
\end{defi}

\begin{defi}
  \label{defi:espmes}
  Un \textbf{espace mesurable} $\left(E, \mathcal{A}\right)$ est un ensemble $E$ muni d'une tribu $\mathcal{A}$.
\end{defi}

\begin{defi}
  \label{defi:ensmes}
  Soit $\left(E, \mathcal{A}\right)$ un espace mesurable. Un ensemble $X\in \mathcal{A}$ est dit $\mathcal{A}$-\textbf{mesurable}.
\end{defi}

\begin{rem}
  Si le contexte est clair (\textit{ie} il n'y a qu'une seule tribu), on pourra omettre $\mathcal{A}$ et parler simplement d'ensemble mesurable
\end{rem}

\begin{rem}
  On remarquera qu'il faut distinguer les \textit{espaces} mesurables (définition \ref{defi:espmes}) des \textit{ensembles} mesurables (définition \ref{defi:ensmes}).
\end{rem}

\begin{defi}
  Soit $E$ un ensemble, et $B$ une collection d'ensembles de $E$.
  La \textbf{tribu engendrée par} $B$ est la plus petite tribu sur $E$ contenant $B$.
  De manière équivalente, c'est l'intersection de toutes les tribus de $E$ contenant $B$.
\end{defi}

\begin{defi}
  \label{defi:Borel}
  Soit $E$ un ensemble muni d'une topologie.
  La \textbf{tribu de Borel} (ou plus simplement, les \textbf{boréliens}) est la tribu engendrée par les ouverts de $E$ (ou de manière équivalente, par les fermés de $E$).
  Elle est notée $\mathcal{B}(E)$.
\end{defi}

\section{Définition de la mesure}

Abordons désormais la notion de mesure.
Un ensemble mesurable doit intuitivement admettre une mesure positive.
De plus, si deux ensembles sont disjoints, il est légitime de demander que la mesure de leur union soit la somme des mesures.

Cela nous amène à définir une mesure ainsi :

\begin{defi}
  Une \textbf{mesure} $\mu$ sur un espace mesurable $(E, \mathcal{A})$ est une application $\mu : \mathcal{A} \longrightarrow \mathbb{R}_+ \cup \{ +\infty \}$ telle que :
  \begin{enumerate}
    \item $\mu(\varnothing) = 0$
    \item Pour une famille au plus dénombrable d'ensembles mesurables $\left(A_i\right)_{i \in I}$,
    $$\mu\left(\bigcup_{i \in I}A_i\right) = \sum_{i \in I} \mu\left(A_i\right)$$
  \end{enumerate}
\end{defi}

\begin{rem}
  Cette deuxième propriété porte le nom de $\sigma$\textbf{-additivité}.
\end{rem}

\begin{rem}
  Si $\mu(\varnothing) \neq 0$, le lecteur pourra vérifier que la mesure de n'importe-quel ensemble mesurable (\textit{a fortiori} du vide) vaut $+\infty$.
  On comprend alors pourquoi on impose que la mesure du vide soit nulle : si ce n'était pas le cas, mesurer n'aurait pour ainsi dire aucun intérêt.
\end{rem}

\begin{defi}
  Un \textbf{espace mesuré} $\left(E, \mathcal{A}, \mu\right)$ est un espace mesurable $\left(E, \mathcal{A}\right)$ muni d'une mesure $\mu$.
\end{defi}

\begin{defi}
  Soit $\left(E, \mathcal{A}, \mu\right)$ un espace mesuré.
  Un ensemble $X \subset E$ est dit $\mu$\textbf{-négligeable} si :
  $$\exists Y \in \mathcal{A}, 
  \left\{
    \begin{array}{c}
      \mu(Y)=0\\
      X \subset Y
    \end{array}
    \right.$$
\end{defi}

\begin{rem}
  \label{rem:completion}
  On remarque qu'un ensemble négligeable n'est pas forcément mesurable.
  Il est cepedant commode d'imposer qu'un ensemble négligeable soit nécessairement mesurable.
  Modifier une mesure et la tribu respective pour arriver à ce résultat porte le nom de \textbf{complétion d'une mesure}.

  Ce processus ne sera pas détaillé ici, mais le lecteur peut néanmois retenir qu'une mesure complétée vérifie l'équivalence suivante : \textbf{un ensemble est négligeable ssi il est mesurable, de mesure nulle}
\end{rem}

\begin{rem}
  On notera que la notion d'ensemble négligeable dépend de la mesure choisie.
  Un ensemble négligeable pour une certaine mesure peut très bien ne pas l'être pour une autre mesure.
\end{rem}

%TODO ajouter la continuité monotone

\section{Exemples importants}
Il existe trois mesures importantes à connaître.

\begin{defi}
  La \textbf{mesure de Lebesgue} $\lambda$ est l'unique mesure sur $\mathbb{R}^n$ qui prolonge la notion de volume.
  Habituellement, sa tribu de définition est la \textbf{tribu de Lebesgue} ou la \textbf{tribu de Borel}.
\end{defi}

\begin{rem}
  En définissant la mesure de Lebesgue sur la tribu de Borel, elle ne serait en fait pas complète (voir \ref{rem:completion}).
  La tribu de Lebesgue (notée $\mathcal{L}(\mathbb{R}^n)$) est alors définie comme étant la plus petite tribu de $\mathbb{R}^n$ permettant à cette mesure d'être complète.
  
  On ne retiendra pas plus de détails sur la tribu de Lebesgue, mais il est bon de retenir qu'\textbf{un élément Borel-mesurable est Lebesgue-mesurable} (la réciproque est fausse).
\end{rem}

\begin{rem}
  On remarque immédiatement que :
  \begin{itemize}
    \itemb lorsque $n=1$, cette mesure prolonge la notion de longueur vue en introduction du chapitre.
    \itemb lorsque $n=2$, cette mesure prolonge la notion de surface.
    \itemb lorsque $n=3$, cette mesure prolonge la notion de volume.
  \end{itemize}

  Par ailleurs, cette mesure sera centrale dans la partie \ref{part:int} concernant les intégrales.
\end{rem}

\begin{defi}
  Soit $\left(E,\mathcal{A}\right)$ un espace mesurable.
  La \textbf{mesure de comptage} $c$ est définie sur cette espace comme suit :
  
  $$
    c : 
  \left\{
  \begin{array}{ccc}
    \mathcal{A} & \longrightarrow & \mathbb{R}_+\cup\left\{+\infty\right\}\\
    A &                                   \longmapsto    &  
      \left\{
      \begin{array}{rcl}
        0 & \text{si} & A = \varnothing\\
        n & \text{si} & A \text{ est fini de cardinal } n\\
        +\infty & \text{si} & A \text{ est infini}
      \end{array}
      \right.
  \end{array}
  \right.
  $$
\end{defi}

\begin{rem}
  Comme son nom l'indique, cette mesure compte les éléments présents dans l'ensemble que l'on mesure.
\end{rem}

\begin{rem}
  Bien que cette mesure soit définie pour un espace mesurable quelconque, elle sera particulièrement utile dans $\left(\mathbb{N},\mathcal{P}\left(\mathbb{N}\right)\right)$ (notamment pour la partie \ref{part:series} concernant les séries).
\end{rem}


\begin{defi}
  Soit $\left(E,\mathcal{A}\right)$ un espace mesurable.
  La \textbf{mesure de Dirac en $x \in E$} $\delta_x$ est définie comme suit :

  $$
    \delta_x : 
  \left\{
  \begin{array}{ccc}
    \mathcal{A} & \longrightarrow & \mathbb{R}_+\\
    A &                                   \longmapsto    &  
      \left\{
      \begin{array}{rcl}
        0 & \text{si} & x \in A\\
        1 & \text{si} & x \notin A
      \end{array}
      \right.
  \end{array}
  \right.
  $$
\end{defi}

\begin{rem}
  La mesure de Dirac et la fonction indicatrice sont reliées de la manière suivante :
  $$
  \forall A \in \mathcal{A}, \forall x \in E, \delta_x(A)=\mathds{1}_A(x)
  $$
\end{rem}

\section{Fonctions mesurables}

Lorsque deux espaces mesurables sont définis, il peut être utile de définir une classe de fonctions qui font bon ménage avec les tribus des espaces mis en jeu.

\begin{defi}
  Soit $\left(E, \mathcal{A}\right)$ et $\left(F, \mathcal{B}\right)$ deux espaces mesurables.
  Une \textbf{fonction } $\mathcal{A}$/$\mathcal{B}$\textbf{-mesurable} est une fonction $f : E \longrightarrow F$ telle que :
  $$ \forall Y \in \mathcal{B}, f^{-1}(Y) \in \mathcal{A} $$
\end{defi}

\begin{rem}
  Cette caractérisation des fonctions mesurables porte le nom de \textbf{critère de l'image réciproque}.
\end{rem}

\begin{rem}
  Cette notation est lourde, et c'est pourquoi elle est souvent simplifiée.
  \begin{itemize}
  \itemb dans le cas où $\mathcal{B}=\mathcal{B}\left(F\right)$ (la tribu d'arrivée est celle des boréliens), on pourra simplifier la notation et dire simplement que :
  $$
  f \text{ est } \mathcal{A}\text{-mesurable}
  $$
  \itemb si, de plus, la tribu de l'espace de départ n'est pas ambigüe, on pourra alors simplement écrire que :
  $$
  f \text{ est mesurable}
  $$
  \end{itemize}
\end{rem}

\begin{rem}
  On remarquera que la notion de fonction mesurable est \textbf{indépendante des mesures choisies}.
  Elle ne dépend que des tribus de l'espace de départ et d'arrivée.
\end{rem}

Voici un lien bien utile entre ensembles et fonctions mesurables :

\begin{prop}
  \label{prop:lienensfctmes}
  Soit $\left(E, \mathcal{A}\right)$ un espace mesurable.
  Soit $A\subset E$.
  Alors $A$ est mesurable ssi $\mathds{1}_A$ est mesurable.
\end{prop}

\begin{rem}
  On remarquera que l'on a bien utilisé le formalisme de la remarque précédente :
  \begin{itemize}
    \itemb la tribu de l'espace d'arrivée est la tribu borélienne de $\mathbb{R}$.
    \itemb la tribu de l'espace de départ est $\mathcal{A}$, non-ambigüe.
  \end{itemize}
\end{rem}

La proposition suivante permet de simplifier considérablement la vérification de la mesurabilité d'une fonction, dans le cas où l'espace d'arrivée est muni de la tribu des boréliens.

Elle découle directement du fait que la tribu des boréliens soit la tribu engendrée par les ouverts :
pour vérifier que le résultat est vrai pour tout élément de la tribu, il suffit de le vérifier sur les éléments qui engendrent la tribu (en l'ocurrence, les ouverts).

\begin{prop}
  \label{prop:simp}
  Soit $\left(E, \mathcal{A}\right)$ et $\left(F, \mathcal{B}(F)\right)$ deux espaces mesurables.
  $f$ est $\mathcal{A}$-mesurable ssi pour tout ouvert $U$ de $F$, $f^{-1}(U) \in \mathcal{A}$.
\end{prop}

\begin{rem}
  En vertu de la définition \ref{defi:Borel}, on peut remplacer dans la proposition \ref{prop:simp} \og ouvert \fg par \og fermé \fg.
\end{rem}

Le résultat suivant est un résultat important concernant la composition des fonctions mesurables :

\begin{prop}
  Soit $\left(E,\mathcal{A}\right)$, $\left(F,\mathcal{B}\right)$ et $\left(G,\mathcal{C}\right)$ trois espaces mesurables,
  $f : E \longrightarrow F$ une fonction $\mathcal{A}$/$\mathcal{B}$-mesurable et $g : F \longrightarrow G$ une fonction $\mathcal{B}$/$\mathcal{C}$-mesurable.
  Alors $g \circ f : E \longrightarrow G$ est une fonction $\mathcal{A}$/$\mathcal{C}$-mesurable
\end{prop}

Enfin, voici une dernière définition qui sera utile dans le chapitre concernant les probabilités :

\begin{defi}
  Une fonction $f:E\longrightarrow F$ est dite \textbf{borélienne} lorsqu'elle est $\mathcal{B}(E)$/$\mathcal{B}(F)$-mesurable.
\end{defi}

\begin{prop}
  Une fonction continue est borélienne.
\end{prop}

\begin{rem}
  Cette proposition est une conséquence directe de la caractérisation de la continuité d'une fonction par l'image réciproque des ouverts.
  Attention, la réciproque de ce résultat est fausse : il existe des fonctions boréliennes qui ne sont pas continues.
\end{rem}

%TODO : exemples de fonctions mesurables

\section{Approximation des fonctions mesurables}

Les fonctions mesurables, en plus de faire bon ménage avec les ensembles mesurables, peuvent être approximées facilement par des fonctions dites \og étagées \fg.
Nous allons commencer par définir cette classe de fonctions, puis nous verrons en quoi consiste cette approximation.

\begin{defi}
  Une \textbf{fonction étagée} est une fonction dont l'image est finie.
\end{defi}

La proposition suivante fournit une décomposition pratique des fonctions étagées :

\begin{prop}
  Soit $f$ une fonction étagée. Alors :
  $$
  \exists N \in \mathbb{N}, \exists \left(A_i\right)_{i \in [\![1,N]\!]} \subset \mathbb{R}^N, \exists \left(y_i\right)_{i \in [\![1,N]\!]} \in \mathbb{R}^N, f = \sum_{i=1}^{N}y_i\mathds{1}_{A_i}
  $$
\end{prop}

\begin{rem}
  Lorsque la fonction étagée que l'on manipule est mesurable, il existe toujours une décomposition de la forme précédente où les $\left(A_i\right)_{i\in[\![1,N]\!]}$ ont le bon goût d'être mesurables.

  C'est une propriété essentielle des fonctions étagées mesurables, qui sera utile dans prochain chapitre.
\end{rem}

\begin{thm}
  \label{thm:decompmes}
  Toute fonction mesurable positive est limite simple d'une suite croissante de fonctions étagées mesurables positives.
\end{thm}

\begin{rem}
  Ici, la croissance d'une suite de fonctions $\left(f_k\right)_{k\in \mathbb{N}}$ est à comprendre dans le sens suivant :
  $$
  \forall k \in \mathbb{N}, \forall x \in E, f_k(x) \leq f_{k+1}(x)
  $$
\end{rem}

Un corollaire existe pour les fonctions mesurables signées, mais il est assez peu utile. Le voici à titre culturel seulement :

\begin{cor}
  Toute fonction mesurable est limite simple d'une suite absolument croissante de fonctions étagées mesurables.
\end{cor}

\begin{rem}
  Ici, l'absolue croissance d'une suite de fonctions $\left(f_k\right)_{k\in \mathbb{N}}$ est à comprendre dans le sens suivant :
  $$
  \forall k \in \mathbb{N}, \forall x \in E, \left|f_k(x)\right| \leq \left|f_{k+1}(x)\right|
  $$
\end{rem}

\end{document}