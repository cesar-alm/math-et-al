\documentclass[../integ-proba.tex]{subfiles}

\begin{document}

Dans ce chapitre, nous allons voir les principaux résultats d'intégration, lorsque l'on muni l'intégrale de Lebesgue de la mesure de Lebesgue.
Dans la pratique, c'est souvent cette catégorie d'intégrale qui sera utilisée, puisqu'elle a un comportement très proche de l'intégrale de Riemann.

On rappelle que, par définition, la mesure de Lebesgue est définie sur l'espace mesurable $\left(\mathbb{R}^n, \mathcal{B}(\mathbb{R}^n)\right)$ ou $\left(\mathbb{R}^n, \mathcal{L}(\mathbb{R}^n)\right)$.
Nos fonctions prendront donc leurs valeurs dans $\mathbb{R}^n$.

Pour faciliter les notations, et souligner cette ressemblance, on notera désormais :

$$
\int_Af(x)\text{d}\lambda(x) = \int_Af(x)\text{d}x
$$

%TODO Fubini Tonelli
%TODO Changement de variables
%TODO IPP

\end{document}