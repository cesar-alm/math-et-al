\documentclass[../integ-proba.tex]{subfiles}

\begin{document}

  \chapter*{Avant-propos}

  Ce document vise à résumer les notions essentielles d'intégration et de probabilités.
  Son objectif est de retracer les cheminements de pensée qui mènent aux résultats essentiels, afin de mieux les retenir.
  Ainsi, il pourra être avantageusement utilisé pour se refamiliariser avec ces notions, au cas-où elles auraient été oubliées.
  Ce travail peut également servir de préambule à une étude plus approfondie du sujet, pour découvrir les fondements de ce domaine des mathématiques.

  Néanmoins, ce document ne saurait se substituer à un ouvrage de référence.
  En effet, par soucis de clarté et de concision, il ne détaille pas toutes les démonstrations.
  Celles-ci sont néanmoins nécessaires pour saisir l'entièreté des résultats énoncés ici.
  De cette manière, \textbf{il faut le lire comme un aide-mémoire ou un résumé, et non comme un cours}.

  \bigskip

  Ce travail se partage en deux grandes parties :
  \begin{itemize}
    \itemb Une première partie commençant par une introduction à la théorie de la mesure, suivie de la construction rapide de l'intégrale de Lebesgue.
    Elle explore ensuite les cas particuliers de l'intégrale :
    \begin{itemize}
      \item munie de la mesure de Lebesgue (l'intégrale \textit{classique}) ;
      \item munie de la mesure de comptage (les séries).
    \end{itemize}
    \itemb Une deuxième partie qui applique la partie précédente à la théorie des probabilités
  \end{itemize}

  \bigskip

  Ce document s'est grandement inspiré du cours \href{https://github.com/boisgera/CDIS}{Calcul Différentiel, Intégral et Stochastique (CDIS)} de l'Ecole des MINES ParisTech.
  L'auteur recommande aux lecteurs intéressés de s'y référer régulièrement.

\end{document}