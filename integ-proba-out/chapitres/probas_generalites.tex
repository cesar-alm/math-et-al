\documentclass[../integ-proba.tex]{subfiles}

\begin{document}

    \chapter{Généralités sur les probabilités}

    \section{Premières définitions}

    Commençons par définir les notions fondamentales de probabilités, à l'aide du formalisme introduit dans la partie précédente.

    \begin{defi}
        On appelle \textbf{espace probabilisable} un espace mesurable, dans le contexte des probabilités.
    \end{defi}

    \begin{defi}
        Soit $\left(\Omega, \mathcal{A}\right)$ un espace probabilisable.
        Une \textbf{mesure de probabilité} (ou \textbf{loi de probabilité}) $\mathbb{P}$ sur $\left(\Omega, \mathcal{A}\right)$ est une mesure sur $\left(\Omega, \mathcal{A}\right)$, à valeurs dans $\left[0,1\right]$, telle que $\mathbb{P}\left(\Omega\right)=1$.
    \end{defi}

    \begin{defi}
        On appelle \textbf{espace probabilisé} un espace probabilisable muni d'une mesure de probabilité.
    \end{defi}

    \section{Fonction de répartition}

    La fonction de répartition est une notion qui ne concerne que les mesures de probabilités sur $\mathbb{R}$.
    La notion existe sur $\mathbb{R}^n$, mais est peu utilisée.
    Dans cette section, on se place donc sur un espace probabilisé $\left( \mathbb{R}, \mathcal{B}(\mathbb{R}), \mathbb{P} \right)$.

    \begin{defi}
        La \textbf{fonction de répartition} de $\mathbb{P}$ est :
        \begin{displaymath}
            F_{\mathbb{P}}:
            \left\{
            \begin{array}{rcl}
                \mathbb{R} & \longrightarrow & [0,1] \\
                x          & \longmapsto     & \mathbb{P}(]- \infty, x])
            \end{array}
            \right.
        \end{displaymath}
    \end{defi}

    \begin{thm}
        La fonction de répartition de $\mathbb{P}$ caractérise $\mathbb{P}$.
    \end{thm}

    \begin{rem}
        En particulier, cela implique que si deux mesures de probabilité ont la même fonction de répartition, alors elles sont égales.
    \end{rem}

    \begin{thm}[\textbf{Conditions nécessaires et suffisantes pour qu'une fonction soit une fonction de répartition}]
        \label{thm:carac_fdrep}
        Soit $F:\mathbb{R} \rightarrow \mathbb{R}$.
        $F$ est la fonction de répartition d'une mesure de probabilité $\mathbb{Q}$ sur $\left( \mathbb{R}, \mathcal{B}(\mathbb{R}) \right)$ si et seulement si les quatre conditions suivantes sont remplies :
        \begin{itemize}
            \itemb $F$ est croissante ;
            \itemb $F$ est continue à droite ;
            \itemb $\lim_{x \to -\infty} F(x) = 0$ ;
            \itemb $\lim_{x \to +\infty} F(x) = 1$.
        \end{itemize}
    \end{thm}

    \begin{prop}[\textbf{Propriétés des fonctions de répartition}]
        La fonction de répartition vérifie les résultats suivants :
        \begin{itemize}
            \itemb $F_\mathbb{P}$ admet une limite à gauche en tout $x \in \mathbb{R}$, notée $F_\mathbb{P}(x^-)$ ;
            \itemb $\mathbb{P}(]x,y]) = F_\mathbb{P}(y) - F_\mathbb{P}(x)$ ;
            \itemb $\mathbb{P}(]x,y[) = F_\mathbb{P}(y^-) - F_\mathbb{P}(x)$ ;
            \itemb $\mathbb{P}([x,y]) = F_\mathbb{P}(y) - F_\mathbb{P}(x^-)$ ;
            \itemb $\mathbb{P}([x,y[) = F_\mathbb{P}(y^-) - F_\mathbb{P}(x^-)$.
        \end{itemize}
    \end{prop}

    \begin{rem}
        En vertu du théorème~\ref{thm:carac_fdrep}, on a bien $F_\mathbb{P}(x) = F_\mathbb{P}(x^+)$, mais nous n'avons aucun résultat équivalent sur la limite à gauche (\textit{ie} sur la continuité à gauche).
        Le corollaire suivant permet néanmoins de caractériser la continuité à gauche.
    \end{rem}

    \begin{cor}[\textbf{Caractérisation de la continuité de la fonction de répartition}]
        Soit $x \in \mathbb{R}$.
        On a $\mathbb{P}(\left\{x\right\}) = F_\mathbb{P}(x) - F_\mathbb{P}(x^-)$.
        En particulier, $\mathbb{P}(\left\{x\right\}) = 0$ ssi $F_\mathbb{P}$ est continue en $x$.
    \end{cor}

    \section{Deux cas particuliers de mesures de probabilités}
    \label{sec:probas_mes_particulieres}

    Il existe un cas particulier où les mesures de probabilité sont simples à exprimer : celui où elles sont définies par une intégrale (cf.\ section~\ref{sec:mes_defi_int}.
    Il permet de grandement simplifier les calculs.

    Ce cas particulier se décline en deux, selon si la mesure considérée est celle de Lebesgue ou celle de comptage.

    \subsection{Mesures de probabilité à densité}
    \begin{defi}
        On se place sur $\mathbb{R}$ (resp. $\mathbb{R}^n$) muni de la tribu borélienne et de la mesure de Lebesgue $\lambda$.
        Soit $f$ une fonction positive et intégrable d'intégrale 1.
        Soit $\mathbb{P}:=f \lambda$.
        Alors $\mathbb{P}$ est une mesure de probabilité, et on dit alors que $\mathbb{P}$ admet une \textbf{densité}.
    \end{defi}

    \begin{rem}
        Notons que la réciproque est fausse : il existe des mesures sur $\mathbb{R}$ (resp. $\mathbb{R}^n$)qui n'admettent pas de densité.
        Par ailleurs, si une mesure de probabilité admet une densité, elle en admet une infinité.
        En effet, si une fonction $f$ convient, alors une autre fonction $g$ qui égale à $f$ presque partout convient également. %todo why
    \end{rem}

    \begin{prop}[\textbf{Propriétés de la fonction de répartition d'une mesure admettant une densité}]
        Soit $\mathbb{P}$ une mesure de probabilité sur $\mathbb{R}$ muni de la tribu borélienne admettant une densité $f$.
        Alors la fonction de répartition $F_\mathbb{P}$ de $\mathbb{P}$ est \textbf{continue}, \textit{ie} $\forall x \in \mathbb{R}, \mathbb{P}(\left\{ x \right\}) = 0$.
        De plus, $F_\mathbb{P}$ est dérivable en tout point $x$ où $f$ est continue, et le cas échéant $F'_\mathbb{P}(x) = f(x)$.
    \end{prop}

    \begin{prop}[\textbf{Une condition suffisante pour admettre une densité (à l'aide de la fonction de répartition)}]
        Soit $\mathbb{P}$ une mesure de probabilité sur $\mathbb{R}$ muni de la tribu borélienne.
        Si $F_\mathbb{P}$ est dérivable, alors $\mathbb{P}$ admet une densité, donnée par $F'$.
    \end{prop}

    % TODO examples


    \subsection{Mesure de probabilité discrètes}
    Le deuxième cas particulier concerne les espaces discrets.
    Sans perte de généralité, on se place uniquement dans le cas de $\mathbb{N}$.

    On commence par énoncer le résultat important suivant :

    \begin{thm}
        Soit $\left(\mathbb{N}, \mathcal{P}(\mathbb{N}), \mathbb{P} \right)$ un espace probabilisé.
        Il existe une unique suite $f:\mathbb{N} \rightarrow \mathbb{R}$ positive et sommable de somme 1 telle que $\mathbb{P} = f c$, où $c$ est la mesure de comptage.
        De plus, $\forall n \in \mathbb{N}, f(n)=\mathbb{P}(\left\{n\right\})$, \textit{ie} :
        \begin{displaymath}
            \forall A \subset \mathbb{N}, \mathbb{P}(A) = \sum_{a \in A} \mathbb{P}(\left\{ a \right\})
        \end{displaymath}
    \end{thm}

    \begin{rem}
        Ce résultat nous prouve que \textbf{les mesures de probabilité discrètes sont caractérisées par leurs valeurs élémentaires)} (\textit{ie} sur les singletons).

        Notons cependant que ce résultat est faux dans le cas général.
        Par exemple, dans le cas des mesures à densité, nous verrons que toutes les valeurs élémentaires sont nulles : il n'y a donc pas unicité.
    \end{rem}

    % TODO examples


\end{document}