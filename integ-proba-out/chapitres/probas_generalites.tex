\documentclass[../integ-proba.tex]{subfiles}

\begin{document}

\section{Premières définitions}

Commençons par définir les notions fondamentales de probabilités, à l'aide du formalisme introduit dans la partie précédente.

\begin{defi}
    On appelle \textbf{espace probabilisable} un espace mesurable, dans le contexte des probabilités.
\end{defi}

\begin{defi}
    Soit $\left(\Omega, \mathcal{A}\right)$ un espace probabilisable.
    Une \textbf{mesure de probabilité} (ou \textbf{loi de probabilité}) $\mathbb{P}$ sur $\left(\Omega, \mathcal{A}\right)$ est une mesure sur $\left(\Omega, \mathcal{A}\right)$, à valeurs dans $\left[0,1\right]$, telle que $\mathbb{P}\left(\Omega\right)=1$.
\end{defi}

\begin{defi}
    On appelle \textbf{espace probabilisé} un espace probabilisable muni d'une mesure de probabilité.
\end{defi}

\section{Variables aléatoires}

Voyons maintenant ce qu'est une variable aléatoire.
C'est une appellation étrange, puisqu'une variable aléatoire n'est en fait ni une variable, ni quelque-chose d'aléatoire.
En réalité, il s'agit simplement d'une fonction qui possède certaines propriétés (notamment la mesurabilité).
Nous nous limitons ici à l'étude des variables aléatoire réelles, des vecteurs aléatoires et des variables aléatoires discrètes.

Dans toute cette section, on se donne un espace probabilisé $\left(\Omega, \mathcal{A}, \mathbb{P}\right)$.
On rappelle que $\mathcal{B}(E)$ désigne la tribu des boréliens de l'espace $E$.

\begin{defi}
    Une \textbf{variable aléatoire réelle} est une fonction mesurable
    $$X:\left(\Omega, \mathcal{A}, \mathbb{P}\right) \longrightarrow \left(\mathbb{R}, \mathcal{B}(\mathbb{R})\right)$$
\end{defi}

\begin{defi}
    Un \textbf{vecteur aléatoire} est une fonction mesurable
    $$X:\left(\Omega, \mathcal{A}, \mathbb{P}\right) \longrightarrow \left(\mathbb{R}^n, \mathcal{B}(\mathbb{R}^n)\right)$$
\end{defi}

\begin{defi}
    Une \textbf{variable aléatoire discrète} est une fonction mesurable
    $$X:\left(\Omega, \mathcal{A}, \mathbb{P}\right) \longrightarrow \left(\mathbb{N}, \mathcal{P}(\mathbb{N})\right)$$
\end{defi}

On peut également introduire la définition suivante, qui ne sera pas utile en pratique, mais qui a le mérite d'unifier les trois définitions précédentes :

\begin{defi}
    Plus généralement, si $E$ est un espace quelconque, une \textbf{variable aléatoire} est une fonction mesurable
    $$X:\left(\Omega, \mathcal{A}, \mathbb{P}\right) \longrightarrow \left(E, \mathcal{B}(E)\right)$$
\end{defi}

\begin{rem}
    On remarquera que, exceptés l'espace d'arrivée et la tribu de l'espace d'arrivée, ces définitions sont identiques.
\end{rem}

\begin{rem}
    \label{rem:vect_alea_idem_var}
    Un vecteur aléatoire ne contient pas \og variable \fg dans son nom, mais c'est bien une variable aléatoire.
    
    Par ailleurs, on vérifie aisément qu'un vecteur aléatoire n'est qu'un vecteur de $n$ variables aléatoires réelles.
    Ainsi, dès que l'on énoncera des résultats d'intégration pour des variables aléatoires réelles, on fera l'économie de ne pas les énoncer pour les vecteurs aléatoires (voir la section \ref{sect:generalisation_integrale_rn})
\end{rem}

\begin{rem}
    On fera attention au fait que :
    \begin{itemize}
        \itemb une variable aléatoire est une fonction qui prend ses valeurs dans $\Omega$ ;
        \itemb une mesure de probabilité est une fonction qui prend ses valeurs dans $\mathcal{A}$.
    \end{itemize}
    Ainsi, si $\omega \in \Omega$ :
    \begin{itemize}
        \itemb écrire $X(\omega)$ a du sens.
        \itemb écrire $\mathbb{P}(\omega)$ n'a pas de sens ($\omega$ n'a aucune chance d'être dans $\mathcal{A}$, puisque ce n'est même pas une partie de $\Omega$).
        \itemb écrire $\mathbb{P}(\{\omega\})$ a du sens dès que $\{\omega\}\in \mathcal{A}$ (\textit{ie} dès que $\{\omega\}$ est mesurable).
        \itemb écrire $X(\{\omega\})$ n'a pas de sens.
    \end{itemize}
\end{rem}

\subsection{Moments d'une variable aléatoire}

Dans toute cette section, on se donne
\begin{itemize}
    \itemb un espace probabilisé $\left(\Omega, \mathcal{A}, \mathbb{P}\right)$
    \itemb une variable aléatoire réelle ou discrète
\end{itemize}

Conformément à la remarque \ref{rem:vect_alea_idem_var}, ces définitions s'étendent immédiatement pour des vecteurs aléatoires.

Cependant, on ne pourra pas donner de sens à ces notions dans le cas général d'une variable aléatoire quelconque (pour la simple et bonne raison que l'on va avoir à faire à l'intégrale de $X$ ; et que celle-ci n'a de sens que si l'espace d'arrivée de $X$ est contenu dans $\mathbb{R}^n$)

\begin{defi}
    Soit $r\in\mathbb{N}$. On dit que $X$ \textbf{admet un moment d'ordre} $r$ lorsque $\omega \mapsto X^r(\omega)$ est $\mathbb{P}$-intégrable.
\end{defi}

\begin{rem}
    On rappelle que l'intégrale de Lebesgue est absolue. Ainsi, cette définition et les assertions suivantes sont équivalentes :
    \begin{itemize}
        \itemb $\omega \mapsto \left|X(\omega)\right|^r$ est intégrable.
        \itemb $\displaystyle \int_\Omega\left|X(\omega)\right|^r\text{d}\mathbb{P}(\omega) < +\infty$
    \end{itemize}
\end{rem}

\begin{defi}
    Soit $r\in\mathbb{N}$. Lorsque $X$ admet un moment d'ordre $r$, on définit son \textbf{moment d'ordre } $r$ comme
    $$m_r := \int_\Omega X^r(\omega)\text{d}\mathbb{P}(\omega)$$
\end{defi}

\begin{rem}
    On vérifie aisément, par la relation mesure-intégrale, que toute variable aléatoire admet un moment d'ordre 0, et que celui-ci vaut toujours 1.
\end{rem}

\begin{defi}
    On dit que $X$ \textbf{admet une espérance} lorsqu'elle admet un moment d'ordre 1.
    Son \textbf{espérance} est alors définie comme son moment d'ordre 1.
    Elle est notée $\mathbb{E}(X)$.
\end{defi}

\begin{rem}
    Par abus de notation, on pourra écrire $\mathbb{E}(X)=+\infty$ lorsque $X$ n'admet pas d'espérance et qu'elle est positive.
    Ainsi, on pourra écrire $\mathbb{E}(X)$ dès que $X$ est positive, qu'elle admette une espérance ou non.
    Vue la place privilégiée qu'occupent les fonctions positives dans la théorie d'intégration de Lebesgue, cette notation ne surprend pas.
    
    Attention cependant, si $X$ n'admet pas d'espérance et n'est pas positive, on s'interdira formellement d'écrire $\mathbb{E}(X)$ (car l'intégrale sous-jacente n'existe peut-être même pas !).
\end{rem}

\begin{rem}
    Ces considérations étant faites, on remarque que :
    \begin{itemize}
        \itemb $X$ admet une espérance ssi $\mathbb{E}(\left| X \right|) < +\infty$ ;
        \itemb $X$ admet un moment d'ordre $r$ ssi $\mathbb{E}(\left| X \right|^r) < +\infty$ ;
        \itemb si $X$ admet un moment d'ordre $r$, alors $m_r = \mathbb{E}(X^r)$.
    \end{itemize}
\end{rem}

\begin{prop}
    \label{prop:moments_cascade}
    Soit $r\in\mathbb{N}$. Si $X$ admet un moment d'ordre $r$, alors elle admet un moment d'ordre $k$ pour tout $k\in[\![0,r]\!]$.
\end{prop}

\begin{defi}
    On dit que $X$ \textbf{admet une variance} lorsqu'elle admet un moment d'ordre 2.
    Sa \textbf{variance} est alors définie comme :
    $$
    \mathbb{V}(X):=\mathbb{E}\left(\left(X - \mathbb{E}\left(X\right)\right)^2\right)=\mathbb{E}\left(X^2\right) - \mathbb{E}\left(X\right)^2
    $$
\end{defi}

\begin{rem}
    Cette définition impose de faire deux remarques :
    \begin{itemize}
        \itemb le théorème qui fournit l'égalité de ces deux expressions s'appelle \textbf{le théorème de König-Huygens}. Sa démonstration est immédiate par linéarité de l'intégrale.
        \itemb d'après la deuxième expression, l'existence de $\mathbb{V}(X)$ est assurée dès que $X$ admet un moment d'ordre 1 et 2.
        Cependant, bien que la définition demande que $X$ admette un moment d'ordre 2, elle ne demande pas qu'elle admette un moment d'ordre 1.
        Nous sommes en fait sauvés par la proposition \ref{prop:moments_cascade} !
    \end{itemize}
\end{rem}

\subsection{Loi d'une variable aléatoire}

Dans toute cette section, on se donne
\begin{itemize}
    \itemb un espace probabilisé $\left(\Omega, \mathcal{A}, \mathbb{P}\right)$
    \itemb une variable aléatoire quelconque $X$ à valeurs dans $\left(E, \mathcal{B}\right)$
\end{itemize}

Ces définitions peuvent être données pour des variables aléatoires quelconques, mais seront utiles en pratique pour les trois types de variables aléatoires que nous avons vues.

\begin{defi}
    La \textbf{loi de probabilité de} $X$ $\mathbb{P}_X$ est définie comme :
    $$
    \mathbb{P}_X=\mathbb{P} \circ X^{-1}
    $$
    Il s'agit d'une loi de probabilité sur $\left(E, \mathcal{B} \right)$
\end{defi}

\begin{defi}
    Soit $\mathbb{Q}$ une loi de probabilité sur $\left(E, \mathcal{B} \right)$.
    On dit que $X$ \textbf{suit la loi} $\mathbb{Q}$ lorsque $\mathbb{P}_X = \mathbb{Q}$.
\end{defi}

\begin{rem}
    Attention, ce n'est pas parce que deux variables aléatoires ont la même loi qu'elles sont égales !
    Voici un contre-exemple. Il utilise cependant des notions qui n'ont pas encore été abordées à ce stade.
    
    Considérons l'espace probabilisé $\Omega := \{\text{beau temps}, \text{mauvais temps}\}$ muni de l'ensemble de ses parties, et de la mesure de probabilité $\mathbb{P}$ uniforme.
    On a donc :
    $$
    \mathbb{P}(\{\text{beau temps}\}) = \mathbb{P}(\{\text{mauvais temps}\}) = \frac{1}{2}
    $$
    Puis posons la variable aléatoire discrète $X$ à valeurs dans $\{0,1\}$, telle que $X(\{\text{beau temps}\})=1$ et $X(\{\text{mauvais temps}\})=0$.
    
    $X$ suit alors la loi uniforme sur l'espace $\{0,1\}$.

    Considérons maintenant l'espace probabilisé $\Omega' := \{\text{temps chaud}, \text{temps froid}\}$ muni de l'ensemble de ses parties, et de la mesure de probabilité $\mathbb{P}'$ uniforme.
    On a donc :
    $$
    \mathbb{P}(\{\text{temps chaud}\}) = \mathbb{P}(\{\text{temps froid}\}) = \frac{1}{2}
    $$
    Puis posons la variable aléatoire discrète $Y$ à valeurs dans $\{0,1\}$, telle que $Y(\{\text{temps chaud}\})=1$ et $Y(\{\text{temps froid}\})=0$.

    $Y$ suit également la loi uniforme sur l'espace $\{0,1\}$.

    Nous avons donc explicité deux variables aléatoires qui ont même loi, mais qui ne sont pas égales (elles n'ont pas le même espace de départ).
\end{rem}

\end{document}