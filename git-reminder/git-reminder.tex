\documentclass[french]{report}
\usepackage{babel}
\usepackage{hyperref}
\usepackage{lmodern}
\usepackage[T1]{fontenc}

\newcommand\itemb{\item[$\bullet$]}

\begin{document}

\title{Aide-mémoire pour \texttt{git}}
\author{César Almecija
\\ étudiant en première année aux MINES ParisTech
\\ \href{mailto:cesar.almecija@mines-paristech.fr}{cesar.almecija@mines-paristech.fr}}

\maketitle

\chapter{Les bases de \texttt{git}}

\section{Quelques éléments de vocabulaire}

Voici quelques notions de base, nécessaires pour comprendre ce document, et plus généralement, l'environnement \texttt{git} :
\begin{itemize}
    \itemb Un \textbf{dépôt} regroupe les fichiers sur lesquels on peut travailler, mais aussi tout l'historique de modification. Il est souvent appelé \textbf{repo}.
    \itemb Un \textbf{commit}, c'est une modification du dépôt.
    Un commit ne peut jamais être supprimé.
    Il est référencé de manière unique par son \textbf{SHA-1}.
    Un commit connaît son ou ses commits parents (\textit{ie} la ou les modifications précédentes dont il est la conséquence directe), mais ne connaît pas ses commits enfants (\textit{ie} les modifications suivantes)
    \itemb \textbf{HEAD} fait référence au \textbf{commit courant}, celui sur lequel on travaille actuellement.
    \itemb Le \textbf{stage} ou \textbf{index} correspond à un ensemble de modifications effectuéees depuis le dernier commit.
    Par défaut, un fichier modifié n'est pas inclus dans le stage, il doit y être ajouté manuellement.
    Les fichiers dans le stage sont ceux qui sont ajoutés au commit suivant.
    \itemb Une \textbf{branche} est un ensemble de commits successifs.
    \itemb Un \textbf{dépôt propre}, c'est un dépôt où les fichiers du dernier commit, les fichiers de l'index et les fichiers sur le disque sont identiques (\textit{ie} un dépôt qui n'a subi aucune modification locale depuis le dernier commit)
    \itemb Le \textbf{remote} correspond au dépôt distant. En général, c'est de là que le projet a été cloné au départ.
    \itemb Un \textbf{merge} correspond à la fusion de deux versions du projet, \textit{ie} dans le cas où le serveur distant et les fichiers locaux sont à des commits différents.
    Dans ce cas, \texttt{git} opère une fusion, qui peut se matérialiser de trois manières différentes :
    % TODO
\end{itemize}

\section{Les commandes de base}

\subsection{Créer un nouveau dépôt}

\begin{itemize}
   \itemb \texttt{git clone url} clone le dépôt distant un dossier portant le nom du projet
    \itemb \texttt{git clone url folder} clone le dépôt existant dans le dossier \texttt{folder}
    \itemb \texttt{git init} initialise \texttt{git} dans ce dossier pour le transformer en dépôt (sans cloner de projet existant)
\end{itemize}

\subsection{Gérer l'index et faire des commits}

\begin{itemize}
    \itemb \texttt{git add file} ajoute le fichier au stage.
    \itemb \texttt{git reset fichier} retire le fichier du stage.
    \itemb \texttt{git commit -m"Message"} fait un commit, avec le message spécifié.
    \itemb \texttt{git status} affiche un résumé de ce qui a été modifié depuis le dernier commit, en distinguant selon si les modifications font partie de l'index ou non.
\end{itemize}

\subsection{Faire le lien avec le dépôt distant}

\begin{itemize}
    \itemb \texttt{git push} envoie les commits effectués en local au serveur distant.
    Si le dépôt distant possède des modifications qui ne sont pas sur le dépôt local, les modifications sont rejetées.
    Il faut alors effectuer un \texttt{git pull}, résoudre les conflits le cas échéant, pour pouvoir envoyer les commits au serveur distant.
    \itemb \texttt{git pull} récupère les commits effectués sur le dépôt distant, et modifie donc les fichiers locaux.
    Si aucun changement n'a été effectué localement, \texttt{git} procède à un \textit{fast-forward}.
    Autrement, \texttt{git} procède à un merge et créé un nouveau commit
    \itemb \texttt{git fetch} récupère les commits effectués sur le dépôt distant, sans modifier les fichiers locaux (les modifications sont stockées dans un autre dossier)
\end{itemize}

\end{document}